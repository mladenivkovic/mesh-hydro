%=============================================
\section{MUSCL-Hancock Method}
%=============================================




%==============================================================
\subsection{Solving Riemann Problems using the MUSCL-Hancock Method}
%==============================================================


\quickfigcap
	{./figures/MUSCL/MUSCL-1D-NO_LIMITER-riemann-sod-shock-1D.png}
	{fig:MUSCL-no-limiter}
	{
		Solution to the sod shock Riemann problem using the MUSCL method without flux limiters.
		The black line is the exact solution.
	}




\quickfigcap
	{./figures/MUSCL/MUSCL-sod_test_modified-EXACT-1D.png}
	{fig:MUSCL-riemann-const}
	{
		Solution to the modified sod test Riemann problem using the MUSCL-Hancock method, an exact Riemann solver and various limiters.
		The black line is the exact solution.
	}



\quickfigcap
	{./figures/MUSCL/MUSCL-sod_test_modified-MINMOD-1D.png}
	{fig:MUSCL-limiter-const-1}
	{
		Solution to the modified sod test Riemann problem using the MUSCL-Hancock method, a minmod limiter, and  various Riemann solvers.
		The black line is the exact solution.
	}

\quickfigcap
	{./figures/MUSCL/MUSCL-sod_test_modified-SUPERBEE-1D.png}
	{fig:MUSCL-limiter-const-2}
	{
		Solution to the modified sod test Riemann problem using the MUSCL-Hancock method, a superbee limiter, and  various Riemann solvers.
		The black line is the exact solution.
	}

\quickfigcap
	{./figures/MUSCL/MUSCL-sod_test_modified-VAN_LEER-1D.png}
	{fig:MUSCL-limiter-const-3}
	{
		Solution to the modified sod test Riemann problem using the MUSCL-Hancock method, a van Leer limiter, and various Riemann solvers.
		The black line is the exact solution.
	}






%=======================================================================
\subsection{Solving Vacuum Riemann Problems using MUSCL-Hancock Method}
%=======================================================================




\quickfigcap
	{./figures/MUSCL/MUSCL-left_vacuum-EXACT-1D.png}
	{fig:MUSCL-left-vacuum-1}
	{
		Left Vacuum Riemann problem for Exact Riemann solver using MUSCL-Hancock method and various limiters
	}
	
\quickfigcap
	{./figures/MUSCL/MUSCL-left_vacuum-HLLC-1D.png}
	{fig:MUSCL-left-vacuum-2}
	{
		Left Vacuum Riemann problem for HLLC Riemann solver using MUSCL-Hancock method and various limiters
	}




\quickfigcap
	{./figures/MUSCL/MUSCL-vacuum_generating-EXACT-1D.png}
	{fig:MUSCL-vacuum-generating-1}
	{
		Vacuum Generating Riemann problem with Exact Riemann solver using MUSCL-Hancock method
	}
	


\quickfigcap
	{./figures/MUSCL/MUSCL-vacuum_generating-HLLC-1D.png}
	{fig:MUSCL-vacuum-generating-2}
	{
		Vacuum Generating Riemann problem with HLLC Riemann solver using MUSCL-Hancock method
	}















%==============================================================
\subsection{Order of Convergence Study}
%==============================================================



\quickfigcap
	{./figures/MUSCL/MUSCL-convergence-dt-500.png}
	{fig:MUSCL-convergence-dt}
	{
		Testing the method's convergence with respect to the time step $\Delta t$ on Sod test initial conditions.
		For a fair comparison, both the cell width $\Delta x$ and the number of steps taken are fixed.
		The points are measurements, the lines are a linear fit, with the slope of the line given in the legend for each Riemann solver used in the legend.
	}


\quickfigcap
	{./figures/MUSCL/MUSCL-convergence-dx.png}
	{fig:MUSCL-convergence-dx}
	{
		Testing the method's convergence with respect to the cell width $\Delta x$ on Sod test initial conditions.
		For a fair comparison, the Courant number $C_{CFL}$ and the total number of steps are fixed.
		The points are measurements, the lines are a linear fit, with the slope of the line given in the legend for each limiter used in the legend.
		The exact Riemann solver has been used.
	}
	
\quickfigcap
	{./figures/MUSCL/MUSCL-convergence-CFL-10000.png}
	{fig:MUSCL-convergence-CFL}
	{
		Testing the method's convergence with respect to the Courant number $C_{CFL}$ on Sod test initial conditions.
		The points are measurements, the lines are just connecting them.
		The slope of $\half$ is plotted for comparison, and to demonstrate the deviation from it.
		The exact Riemann solver has been used.
	}























%==============================================================
\subsection{Conclusions}
%==============================================================


\begin{itemize}

	\item 	The MUSCL method without limiters, fig. \ref{fig:MUSCL-no-limiter}, gives terrible results.
			In fact, in a lot of cases the oscillations would grow too strong and the code would crash.

	\item 	Similar to the piecewise linear advection, the drops around jump discontinuities are now much sharper, i.e. less diffusive, when limiters are applied.
			Compare fig. \ref{fig:MUSCL-riemann-const} to fig. \ref{fig:godunov-4}.

	\item 	The choice of the Riemann solver has no noticeable effect on the solution. 
			See fig. \ref{fig:MUSCL-riemann-const}.
	
	\item 	The effect of the limiter is comparable to how they behave on linear advection. 
			See fig. \ref{fig:MUSCL-limiter-const-1} - \ref{fig:MUSCL-limiter-const-3}.


	\item 	If vacuum already exists, the method handles it reasonably well (fig \ref{fig:MUSCL-left-vacuum-1}, \ref{fig:MUSCL-left-vacuum-2}) compared with the other two finite volume methods.
			If there is vacuum generating conditions (fig \ref{fig:MUSCL-vacuum-generating-1}, \ref{fig:MUSCL-vacuum-generating-2}), things go down the drain. 
			I had to use $C_{cfl} = 0.1$ to get results like in those figure.
			Clearly the method is not TVD in that case, because new peaks arise.
			But I don't understand exactly why.
			
			

	\item \textbf{Order of Convergence}
	
		\begin{itemize}
			
			\item 	The time step dependence (fig. \ref{fig:MUSCL-convergence-dt}) is comparable to the results of piecewise linear advection, fig. \ref{fig:advection-convergence-dt-step}. 
					
			\item 	For the cell width dependence (fig. \ref{fig:MUSCL-convergence-dx}), we get a remarkable slope of $\approx 1$ for all solvers.

			
			\item 	For the $C_{cfl}$ dependence, it is essentially a power law (straight line in log space) for all $C_{CFL}$, which is telling us that the diffusion which is $\propto (1 - C_{CFL})$ for first order methods is strongly reduced.
		\end{itemize}
\end{itemize}