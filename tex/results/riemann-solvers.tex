%=========================================
\section{Riemann Solvers}
%=========================================



%=========================================
\subsection{Approximate Riemann Solutions}
%=========================================






\quickfigcap
	{./figures/riemann/riemann-approximate-sod_test-0.100.png}
	{fig:riemann-approximate-sod-test}
	{
		The Sod test solved using the exact and approximate Riemann solvers
	}
	

\quickfigcap
	{./figures/riemann/riemann-approximate-sod_test-0.200.png}
	{fig:riemann-approximate-sod-test-later}
	{
		The Sod test solved using the exact and approximate Riemann solvers, at a later time
	}
	




\quickfigcap
	{./figures/riemann/riemann-approximate-sod_test_modified-0.100.png}
	{fig:riemann-approximate-sod-test-modified}
	{
		The modified Sod test solved using the exact and approximate Riemann solvers
	}
	

\quickfigcap
	{./figures/riemann/riemann-approximate-sod_test_modified-0.200.png}
	{fig:riemann-approximate-sod-test-modified-later}
	{
		The modified Sod test solved using the exact and approximate Riemann solvers, at a later time
	}
	



\quickfigcap
	{./figures/riemann/riemann-approximate-left_blast_wave-0.006.png}
	{fig:riemann-approximate-left-blast-wave}
	{
		The left blast wave solved using the exact and approximate Riemann solvers
	}
	

\quickfigcap
	{./figures/riemann/riemann-approximate-left_blast_wave-0.013.png}
	{fig:riemann-approximate-left-blast-wave-later}
	{
		The left blast wave solved using the exact and approximate Riemann solvers, at a later time
	}
	





\quickfigcap
	{./figures/riemann/riemann-approximate-right_blast_wave-0.006.png}
	{fig:riemann-approximate-right-blast-wave}
	{
		The left blast wave solved using the exact and approximate Riemann solvers
	}
	

\quickfigcap
	{./figures/riemann/riemann-approximate-right_blast_wave-0.013.png}
	{fig:riemann-approximate-right-blast-wave-later}
	{
		The left blast wave solved using the exact and approximate Riemann solvers, at a later time
	}
	




	
	
	
	
%=========================================
\subsection{Conclusions}
%=========================================
	
	
\begin{itemize}

	\item 	The TSRS solver may have trouble with rarefactions, e.g. figs \ref{fig:riemann-approximate-sod-test} - \ref{fig:riemann-approximate-sod-test-modified-later}.
			Which is to be expected, since it assumes that we have two shocks present.
			It deals much better with shocks, e.g. figs \ref{fig:riemann-approximate-left-blast-wave} - \ref{fig:riemann-approximate-right-blast-wave-later}.
	
	\item 	The TRRS solver may have trouble with shocks, e.g. figs \ref{fig:riemann-approximate-left-blast-wave} - \ref{fig:riemann-approximate-right-blast-wave-later}.
			Which is to be expected, since it assumes that we have two rarefactions present.
			It deals much better with rarefactions, e.g. figs \ref{fig:riemann-approximate-sod-test} - \ref{fig:riemann-approximate-sod-test-modified-later}.
			
			
	\item 	It looks like the results get worse over time.
			Compare figs \ref{fig:riemann-approximate-sod-test} vs \ref{fig:riemann-approximate-sod-test-later}, \ref{fig:riemann-approximate-sod-test-modified} vs \ref{fig:riemann-approximate-sod-test-modified-later}, etc.
			But recall that for the Riemann solver, we only solve the solution once, and then sample the solution for a given $x$ and $t$.
			So once the four regions are determined initially, all the solver does is ``smear them out'' while sampling at a later time $t$.
	
\end{itemize}
