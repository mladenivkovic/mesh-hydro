\section{Advection}



%=======================================
\subsection{Piecewise Constant}
%=======================================




\quickfigcap
	{figures/advection/advection-1D-pwconst-density-only-overplotted.png}
	{fig:advection-pwconst-four-shapes-fixed-positive-vel}
	{
		Piecewise constant advection with positive fixed global velocity $v_x = 1$. $C_{CFL} = 0.9$,  $nx = 500$
	}

\quickfigcap
	{figures/advection/advection-1D-pwconst-LOWCFL.png}
	{fig:advection-pwconst-four-shapes-fixed-positive-vel-LOWCFL}
	{
		Piecewise constant advection with positive fixed global velocity $v_x = 1$. 
		$C_{CFL} = 0.1$,  $nx = 500$
	}



\quickfigcap
	{figures/advection/advection-1D-CFL1-density-only-overplotted.png}
	{fig:advection-pwconst-four-shapes-fixed-positive-vel-CFL1}
	{
		Piecewise constant advection with positive fixed global velocity $v_x = 1$. 
		$C_{CFL} = 1.0$, $nx = 500$
	}


\quickfigcap
	{figures/advection/advection-1D-pwconst-negative-velocity-density-only-overplotted.png}
	{fig:advection-pwconst-four-steps-fixed-negative-vel}
	{
		Piecewise constant advection with NEGATIVE fixed global velocity $v_x = -1$. 
		$C_{CFL} = 0.9$, $nx = 100$
	}









\quickfigcap
	{figures/advection/advection-2D-step-100-ux-ADVECTION_PWCONST-NO_LIMITER-2D-0001-density-only-NO-STRANG-SPLITTING.png}
	{fig:advection-pwconst-step-2D-fixed-positive-vel-x-no-strang}
	{
		Piecewise constant advection with fixed global velocity $v_x = 1, v_y = 0$. $C_{CFL} = 0.9$, $nx = 100$. 
		ICs were a step function.
		This is 2D \textbf{without} Strang splitting, done naively.
	}
	

\quickfigcap
	{figures/advection/advection-2D-step-100-ux-ADVECTION_PWCONST-NO_LIMITER-2D-0001-density-only.png}
	{fig:advection-pwconst-step-2D-fixed-positive-vel-x}
	{
		Piecewise constant advection with fixed global velocity $v_x = 1, v_y = 0$. $C_{CFL} = 0.9$, $nx = 100$. 
		ICs were a step function.
		\textbf{With} Strang splitting.
	}





	
	


\quickfigcap
	{figures/advection/advection-2D-step-100-uy-ADVECTION_PWCONST-NO_LIMITER-2D-0001-density-only-NO-STRANG-SPLITTING.png}
	{fig:advection-pwconst-step-2D-fixed-positive-vel-y-no-strang}
	{
		Piecewise constant advection with fixed global velocity $v_x = 0,  v_y = 1$. 
		$C_{CFL} = 0.9$, $nx = 100$, $t=1$. 
		ICs were a step function.
		\textbf{Without} Strang splitting, done naively.
	}





\quickfigcap
	{figures/advection/advection-2D-step-100-uy-ADVECTION_PWCONST-NO_LIMITER-2D-0001-density-only.png}
	{fig:advection-pwconst-step-2D-fixed-positive-vel-y}
	{
		Piecewise constant advection with fixed global velocity $v_x = 0,  v_y = 1$. 
		$C_{CFL} = 0.9$, $nx = 100$, $t=1$. 
		ICs were a step function.
		\textbf{With} Strang splitting.
	}







\quickfigcap
	{figures/advection/advection-2D-step-100-ux-uy-ADVECTION_PWCONST-NO_LIMITER-2D-0001-density-only-NO-STRANG-SPLITTING.png}
	{fig:advection-pwconst-step-2D-fixed-positive-vel-xy-no-strang}
	{
		Piecewise constant advection with fixed global velocity $v_x = v_y = 1$.
		$C_{CFL} = 0.9$, $nx = 100$, $t=1$. 
		ICs were a step function. 
		This is 2D \textbf{without} Strang splitting, done naively.
	}

\quickfigcap
	{figures/advection/advection-2D-step-100-ux-uy-ADVECTION_PWCONST-NO_LIMITER-2D-0001-density-only.png}
	{fig:advection-pwconst-step-2D-fixed-positive-vel-xy}
	{	
		Piecewise constant advection with fixed global velocity $v_x = v_y = 1$. 
		$C_{CFL} = 0.9$, $nx = 100$, $t=1$. 
		ICs were a step function.
		This is 2D \textbf{with} Strang splitting.
	}








%=======================================
\subsection{Piecewise Linear}
%=======================================






\quickfigcap
	{figures/advection/advection-1D-pwlin-density-only-overplotted.png}
	{fig:advection-pwlin-four-shapes-fixed-positive-vel}
	{
		Piecewise linear advection with positive fixed global velocity $v_x = 1$. 
		$C_{CFL} = 0.9$,  $nx = 100$
	}





\quickfigcap
	{figures/advection/advection-2D-step-100-ux-ADVECTION_PWLIN-NO_LIMITER-2D-0001-density-only-NO-STRANG-SPLITTING.png}
	{fig:advection-pwlin-step-2D-fixed-positive-vel-x-no-strang}
	{
		Piecewise linear advection with fixed global velocity $v_x = 1, v_y = 0$. 
		$C_{CFL} = 0.9$, $nx = 100$. 
		ICs were a step function.
		\textbf{Without} Strang splitting.
	}







\quickfigcap
	{figures/advection/advection-2D-step-100-ux-ADVECTION_PWLIN-NO_LIMITER-2D-0001-density-only.png}
	{fig:advection-pwlin-step-2D-fixed-positive-vel-x}
	{
		Piecewise linear advection with fixed global velocity $v_x = 1, v_y = 0$. 
		$C_{CFL} = 0.9$, $nx = 100$. 
		ICs were a step function.
		\textbf{With} Strang splitting.
	}







\quickfigcap
	{figures/advection/advection-2D-step-100-uy-ADVECTION_PWLIN-NO_LIMITER-2D-0001-density-only-NO-STRANG-SPLITTING.png}
	{fig:advection-pwlin-step-2D-fixed-positive-vel-y-no-strang}
	{
		Piecewise linear advection with fixed global velocity $v_x = 0,  v_y = 1$. 
		$C_{CFL} = 0.9$, $nx = 100$. 
		ICs were a step function.
		\textbf{Without} Strang splitting.
	}
	
	





\quickfigcap
	{figures/advection/advection-2D-step-100-uy-ADVECTION_PWLIN-NO_LIMITER-2D-0001-density-only.png}
	{fig:advection-pwlin-step-2D-fixed-positive-vel-y}
	{
		Piecewise linear advection with fixed global velocity $v_x = 0,  v_y = 1$. 
		$C_{CFL} = 0.9$, $nx = 100$. 
		ICs were a step function.
		\textbf{With} Strang splitting.
	}
	
	
	



\quickfigcap
	{figures/advection/advection-2D-step-100-ux-uy-ADVECTION_PWLIN-NO_LIMITER-2D-0001-density-only-NO-STRANG-SPLITTING.png}
	{fig:advection-pwlin-step-2D-fixed-positive-vel-xy-no-strang}
	{	
		Piecewise linear advection with fixed global velocity $v_x = v_y = 1$. $C_{CFL} = 0.9$, $nx = 100$. 
		ICs were a step function. 
		Note that despite of the strong oscillations, the total density is conserved! 
		(Unless the oscillations get too big for floats to handle).
		\textbf{Without} Strang splitting.
	}



\quickfigcap
	{figures/advection/advection-2D-step-100-ux-uy-ADVECTION_PWLIN-NO_LIMITER-2D-0001-density-only.png}
	{fig:advection-pwlin-step-2D-fixed-positive-vel-xy}
	{	
		Piecewise linear advection with fixed global velocity $v_x = v_y = 1$. $C_{CFL} = 0.9$, $nx = 100$. 
		ICs were a step function. 
		Note that despite of the strong oscillations, the total density is conserved! 
		(Unless the oscillations get too big for floats to handle)
		\textbf{With} Strang splitting.
	}










\quickfigcap
	{figures/advection/advection-2D-step-100-ux-uy-ADVECTION_PWLIN-NO_LIMITER-2D-0001-density-only-LOWCFL.png}
	{fig:advection-pwlin-step-2D-fixed-positive-vel-xy}
	{
		Piecewise linear advection with fixed global velocity $v_x = v_y = 1$. 
		$C_{CFL} = 0.1$, $nx = 100$. 
		ICs were a step function. 
		Note the lower CFL condition compared to fig \ref{fig:advection-pwlin-step-2D-fixed-positive-vel-xy-no-strang}.
		\textbf{Without} Strang splitting.
	}


















\subsection{Order of Convergence}




\quickfigcap
	{./figures/limiters/limiter_accuracy_dx-gaussian.png}
	{fig:advection-convergence-dx-gaussian}
	{
		Convergence Study with respect to cell size $\Delta x$ using the $L1$ norm.
		Points are measurements, the straight lines are fitted curves, with their slope given in the legend.
		For an accurate comparison, the simulations are stopped after the same number of steps.
		This way, the $C_{cfl}$ can also be kept constant throughout the different $\Delta x$.
		Initial conditions was a Gaussian profile.
	}

\quickfigcap
	{./figures/limiters/limiter_accuracy_dx-step.png}
	{fig:advection-convergence-dx-step}
	{
		Convergence Study with respect to cell size $\Delta x$ using the $L1$ norm.
		Points are measurements, the straight lines are fitted curves, with their slope given in the legend.
		For an accurate comparison, the simulations are stopped after the same number of steps.
		This way, the $C_{cfl}$ can also be kept constant throughout the different $\Delta x$.
		Initial conditions was a step function.
	}
	



\quickfigcap
	{./figures/limiters/limiter_accuracy_dt-gaussian-500.png}
	{fig:advection-convergence-dt-gaussian}
	{
		Convergence Study with respect to time step size $\Delta t$ using the $L1$ norm.
		Points are measurements, the straight lines are fitted curves, with their slope given in the legend.
		For an accurate comparison, the simulations are stopped after the same number of steps, and $\Delta x$ is kept constant.
		Highest $\Delta t$ corresponds to $C_{cfl} = 0.001$.
		Initial conditions was a Gaussian profile.
	}


\quickfigcap
	{./figures/limiters/limiter_accuracy_dt-step-500.png}
	{fig:advection-convergence-dt-step}
	{
		Convergence Study with respect to time step size $\Delta t$ using the $L1$ norm.
		Points are measurements, the straight lines are fitted curves, with their slope given in the legend.
		For an accurate comparison, the simulations are stopped after the same number of steps, and $\Delta x$ is kept constant.
		Highest $\Delta t$ corresponds to $C_{cfl} = 0.001$.
		Initial conditions was a step function.
	}
	



\quickfigcap
	{./figures/limiters/limiter_accuracy_CFL-gaussian-500.png}
	{fig:advection-convergence-CFL-gaussian}
	{
		Convergence Study with respect to the Courant number $C_{cfl}$ using the $L1$ norm.
		Points are measurements, the straight lines just connect the dots.
		For an accurate comparison, the simulations are stopped after the same number of steps, and $\Delta x$ is kept constant.
		Initial conditions was a Gaussian profile.
	}


\quickfigcap
	{./figures/limiters/limiter_accuracy_CFL-step-500.png}
	{fig:advection-convergence-CFL-step}
	{
		Convergence Study with respect to the Courant number $C_{cfl}$ using the $L1$ norm.
		Points are measurements, the straight lines just connect the dots.
		For an accurate comparison, the simulations are stopped after the same number of steps, and $\Delta x$ is kept constant.
		Initial conditions was a step function.
	}
	



















\subsection{Conclusions} \label{chap:advection-conclusions}

\begin{itemize}


	\item Advection is diffusive (fig \ref{fig:advection-pwconst-four-shapes-fixed-positive-vel}).
	
	
	
	
	\item It is diffusive even if $C_{CFL} = 1$! (fig \ref{fig:advection-pwconst-four-shapes-fixed-positive-vel-CFL1}). 
            This is most probably because of roundoff/float errors. For high $t$, the whole shape moves a bit to the right even.
 
 
            
    \item Using a lower CFL number leads to stronger diffusion. 
    	Compare figs. \ref{fig:advection-pwconst-four-shapes-fixed-positive-vel} and \ref{fig:advection-pwconst-four-shapes-fixed-positive-vel-LOWCFL}. Why?
    
    	We are solving the 1D advection equation with $c$ being a constant velocity:
    	\begin{equation}
    		\frac{\partial u}{\partial t} + c\frac{\partial u}{\partial x} = 0 \label{eq:advection-1D}
    	\end{equation}
    
    	Discretising this equation, we get (using an explicit time scheme and upwind differencing):
    	
    	\begin{equation}
    		\frac{u_i^{n+1} - u_i^n}{\Delta t}+\frac{u_i^n-u_{i-1}^{n}}{\Delta x} = 0 \label{eq:advection-discretized}
    	\end{equation}
    	
    	This is however not an exact expression, but only an approximate one.
    	If we use a Taylor expansion
    	\begin{align}
    		u_i^{n+1} &= u_i^n + \Delta t \frac{\partial u}{\partial t}+\frac{\Delta t^2}{2}\frac{\partial^2 u }{\partial t^2}+O(\Delta t^3) \\
    		u_{i-1}^{n} &= u_{i}^n - \Delta x \frac{\partial u}{\partial x}+\frac{\Delta x^2}{2}\frac{\partial^2 u }{\partial x^2}+O(\Delta x^3)
    	\end{align}

		and insert it into eq \ref{eq:advection-discretized}, we get (neglecting third order terms from now on)
		
		\begin{align}
			& \frac{\partial u}{\partial t} + c\frac{\partial u}{\partial x} + \frac{1}{2}\Delta t \frac{\partial^2 u }{\partial t^2} - c \frac{\Delta x}{2} \frac{\partial^2 u }{\partial x^2} = 0
		\end{align}
		\begin{align}
			\Rightarrow \frac{\partial u}{\partial t} + c\frac{\partial u}{\partial x} &= 
				-\frac{1}{2}\Delta t\frac{\partial^2 u }{\partial t^2} + c \frac{\Delta x}{2} \frac{\partial^2 u }{\partial x^2} \\
				&= 0 + Err \\
			Err &= -\frac{1}{2}\Delta t\frac{\partial^2 u }{\partial t^2} + c \frac{\Delta x}{2} \frac{\partial^2 u }{\partial x^2}
		\end{align}
		
		which is the advection equation \ref{eq:advection-1D} plus some error term.
		
		Now using eq. \ref{eq:advection-1D} we find:
		
		\begin{align}
    		& \frac{\partial u}{\partial t} + c\frac{\partial u}{\partial x} = 0 \\
    		1) ~~~ & \frac{\partial^2 u}{\partial t2} + c\frac{\partial^2 u}{\partial t\partial x} = 0 \\
    		2) ~~~ & \frac{\partial u}{\partial t} = -c \frac{\partial u}{\partial x} \\
    		\Rightarrow ~~ 3) ~~~ & \frac{\partial^2 u}{\partial t^2} - c^2 \frac{\partial^2 u}{\partial x^2} = 0
		\end{align}
		
		This gives us for the error term:
		
		\begin{align}
			Err &= 
				-\frac{1}{2}\Delta t\frac{\partial^2 u }{\partial t^2} + c \frac{\Delta x}{2} \frac{\partial^2 u }{\partial x^2} \\
				&= -\frac{c^2 \Delta t}{2}\frac{\partial^2 u }{\partial x^2} + c \frac{\Delta x}{2} \frac{\partial^2 u }{\partial x^2} \\
				&= \frac{c\Delta x}{2} \left(1 - \frac{c \Delta t}{\Delta x} \right) \frac{\partial^2 u }{\partial x^2}
		\end{align}
		
		Inserting the CFL condition:
		\begin{equation}
			\frac{c \Delta t_{max}}{\Delta x} = C_{cfl} \leq 1
		\end{equation}
		
		we obtain:

		\begin{align}
			Err &= \frac{c\Delta x}{2} \left(1 - C_{cfl} \right) \frac{\partial^2 u }{\partial x^2} \label{eq:advection-diffusivity-error}
		\end{align}		
		
		
		The second derivative in space is characteristic for diffusion.
		But you can immediately see that the diffusion term (coefficient) depends on $C_{cfl}$, and indeed increases with decreasing $C_{cfl}$!
		
		
		
	\item 2D advection:
	

		For advection only in x or y direction, respectively, the method reduces to the one dimensional case, and the results are identical when using Strang splitting or the naive case.
		Compare figs \ref{fig:advection-pwconst-step-2D-fixed-positive-vel-x-no-strang} vs \ref{fig:advection-pwconst-step-2D-fixed-positive-vel-x}, and \ref{fig:advection-pwconst-step-2D-fixed-positive-vel-y-no-strang} vs \ref{fig:advection-pwconst-step-2D-fixed-positive-vel-y}.
		But when we have both $v_x$ and $v_y$ be non-zero, the naive method (i.e. without Strang splitting, where we just extend the 1D method to 2D and don't think about it) starts getting ``stripes'' perpendicular to the diagonal along which it is being advected along (fig \ref{fig:advection-pwconst-step-2D-fixed-positive-vel-xy-no-strang} vs \ref{fig:advection-pwconst-step-2D-fixed-positive-vel-xy}).
		The reason is that a) the upwinding is not complete, i.e. the value $u_{i-1, j-1}$ along the diagonal in the naive case never gets properly advected to $u_{i, j}$, and b) the method is diffusive, so the diffused material from $u_{i-1, j}$ and $u_{i, j-1}$ come together in that cell, messing everything up.
		Letting the code run for longer times actually leads to stripe-like instabilities. 
		See also fig \ref{fig:advection-pwlin-step-2D-fixed-positive-vel-xy-no-strang}.



			
		
		
		
		


    \item The piecewise linear scheme can/will introduce oscillations around sharp edges (fig \ref{fig:advection-pwlin-four-shapes-fixed-positive-vel}. 
            The oscillations can go into the negative regime. 
            Even though it's unphysical, the total density content remains constant!
            This is because the scheme is fundamentally conservative.
            
            
            
    \item On the order of convergence:
     
     	
		\begin{itemize}
  			\item Dependence on $\Delta x$:
				
				For the smooth Gaussian profile IC: The results are nicely as expected, fig. \ref{fig:advection-convergence-dx-gaussian}.
				The piecewise constant advection is of $\order(\Delta x)$, piecewise linear is of $\order(\Delta x^2)$.
				Remember that the order of convergence computation on paper gives you only the upper boundary of the error, so a faster convergence is possible, but shouldn't be greater than 1 order than predicted, which we have here.
				
				
				For the step function in fig \ref{fig:advection-convergence-dx-step} we see as expected that the convergence rate drops to $\order(\Delta x)$ because that's what slope limiters do.
				So then why is the unlimited linear method also dropping to first order?
				The computations we did on paper for the order estimate assume a smooth initial condition, which we don't have here. 
				It turns out having a discontinuity drops your order of convergence.
				
			
			
			\item Dependence on $\Delta t$:
			
				Fixing the time step $\Delta t$ is essentially the same as fixing the Courant number $C_{cfl}$ for an already fixed velocity $u$ and grid spacing $\Delta x$.
				For the order of convergence to be measured w.r.t. $\Delta t$, we must start with a $\Delta t$ that implies a very low $C_{cfl}$.
				The reason is that $C_{cfl}$ determines the amplitude of the diffusivity, see eqn. \ref{eq:advection-diffusivity-error} and fig. \ref{fig:advection-convergence-CFL-gaussian}
				So for a good comparison, we need to start with a low enough $C_{cfl}$ such that the difference in diffusivity is negligible.
				Otherwise, you don't see the power law that emerges.
				
				In the following analysis, let's focus on piecewise constant (first order) advection alone.
				I don't have the theory present for the piecewise linear scheme to back up my findings. 
				What makes things more difficult is that for the piecewise linear advection, $\Delta t$ also enters the computation of the fluxes between the cells, thus also affects the spatial component. 
				It is not trivial to separate between the $\Delta x$ and $\Delta x$ dependence in these cases.
				
				For the Gaussian profile, fig. \ref{fig:advection-convergence-dt-gaussian}, we get a nice power law with slope $1$, as expected.
				(The piecewise linear methods get close to $\order(\Delta t ^ {\half})$. 
				Maybe because they start developing jump discontinuities?
				Compare fig \ref{fig:limiters-advection}.)
				For the step profile of the piecewise constant advection, fig. \ref{fig:advection-convergence-dt-step}, we get $\order(\Delta t ^ {\half})$.
				The reason behind it is that the convergence analysis in theory assumes a smooth initial condition, such that we can use derivatives and Taylor-expansions.
				The step function is not smooth though.
				It can be shown that for a jump discontinuity, the error goes as
				
				\begin{equation*}
				     	Err \propto \sqrt{t}
				\end{equation*}
				
				So if we have $N$ steps of equal size $\Delta t)$, if we keep $N$ constant, we get
				\begin{equation}
				     	Err \propto \sqrt{\Delta t}
				\end{equation}
				
				See \cite{leveque_2002}, chapter 8.6 for details.
				
				which is exactly what we see in  fig. \ref{fig:advection-convergence-dt-step}.
				
			
			\item  Dependence on $C_{CFL}$:
				
				Tweaking around $C_{CFL}$ for a fixed $nx$ and advection velocity $u$ is essentially the same as tweaking $\Delta t$.
				For $C_{cfl}$ comparable to 1 however we see the effects of the increased diffusivity, as described by equation \ref{eq:advection-diffusivity-error}.
				Indeed, measuring the convergence of $0.1 \leq C_{cfl} \leq 0.9 $ in figs. \ref{fig:advection-convergence-CFL-gaussian} and \ref{fig:advection-convergence-CFL-step} shows that it doesn't behave like a power law at all.
			
		\end{itemize}


     
     
\end{itemize}


















%======================================
\subsection{Future Debugging Hints}
%======================================




\quickfigcap
	{./figures/advection/too_big_cfl-0000-density-only.png}
	{fig:advection-debugging-IC}
	{Initial conditions used to demonstrate debugging hints. $u = 1$.}


\subsubsection{When you're using downwind differencing}

Downwind differencing is unphysical and violently unstable.
Note how the wave goes in the wrong direction!


\quickfigcap
	{./figures/advection/downwind_differencing_step-0001-density-only.png}
	{fig:advection-debugging-downwind-1}
	{Downwind differencing after 1 step.}

\quickfigcap
	{./figures/advection/downwind_differencing_step-0002-density-only.png}
	{fig:advection-debugging-downwind-2}
	{Downwind differencing after 2 steps.}

\quickfigcap
	{./figures/advection/downwind_differencing_step-0003-density-only.png}
	{fig:advection-debugging-downwind-3}
	{Downwind differencing after 3 steps.}

\quickfigcap
	{./figures/advection/downwind_differencing_step-0005-density-only.png}
	{fig:advection-debugging-downwind-5}
	{Downwind differencing after 5 steps.}

\quickfigcap
	{./figures/advection/downwind_differencing_step-0010-density-only.png}
	{fig:advection-debugging-downwind-10}
	{Downwind differencing after 10 steps.}








\subsubsection{When the CFL condition is violated}


\quickfigcap
	{./figures/advection/too_big_cfl-0001-density-only.png}
	{fig:advection-debugging-downwind-1}
	{Violating the CFL condition (here $C_{cfl} = 1.5$) after 1 step.}

\quickfigcap
	{./figures/advection/too_big_cfl-0002-density-only.png}
	{fig:advection-debugging-downwind-2}
	{Violating the CFL condition (here $C_{cfl} = 1.5$) after 2 steps.}

\quickfigcap
	{./figures/advection/too_big_cfl-0003-density-only.png}
	{fig:advection-debugging-downwind-3}
	{Violating the CFL condition (here $C_{cfl} = 1.5$) after 3 steps.}

\quickfigcap
	{./figures/advection/too_big_cfl-0005-density-only.png}
	{fig:advection-debugging-downwind-5}
	{Violating the CFL condition (here $C_{cfl} = 1.5$) after 5 steps.}

\quickfigcap
	{./figures/advection/too_big_cfl-0010-density-only.png}
	{fig:advection-debugging-downwind-10}
	{Violating the CFL condition (here $C_{cfl} = 1.5$) after 10 steps.}

