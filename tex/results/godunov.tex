%=============================================
\section{Godunov's Method}
%=============================================




%==============================================================
\subsection{Solving Riemann Problems using Godunov's Method}
%==============================================================


\quickfigcap
	{./figures/godunov/GODUNOV-left_blast_wave-1D.png}
	{fig:godunov-1}
	{
		Solution to the left blast wave 
		Riemann problem using Godunov's method and various Riemann solvers.
		The black line is the exact solution.
	}

\quickfigcap
	{./figures/godunov/GODUNOV-right_blast_wave-1D.png}
	{fig:godunov-2}
	{
		Solution to the right blast wave 
		Riemann problem using Godunov's method and various Riemann solvers.
		The black line is the exact solution.
	}

\quickfigcap
	{./figures/godunov/GODUNOV-sod_test-1D.png}
	{fig:godunov-3}
	{
		Solution to the sod test
		Riemann problem using Godunov's method and various Riemann solvers.
		The black line is the exact solution.
	}

\quickfigcap
	{./figures/godunov/GODUNOV-sod_test_modified-1D.png}
	{fig:godunov-4}
	{
		Solution to the modified sod test
		Riemann problem using Godunov's method and various Riemann solvers.
		The black line is the exact solution.
	}

	




%=======================================================================
\subsection{Solving Vacuum Riemann Problems using Godunov's Method}
%=======================================================================


%=======================================================================
\subsubsection{Left Vacuum State}
%=======================================================================



\quickfigcap
	{./figures/godunov/GODUNOV-left_vacuum-1D-0.001.png}
	{fig:godunov-left-vacuum-1}
	{
		Left Vacuum Riemann problem at $t = 0.001$ for different Riemann solvers using Godunov's method
	}
	
\quickfigcap
	{./figures/godunov/GODUNOV-left_vacuum-1D-0.010.png}
	{fig:godunov-left-vacuum-2}
	{
		Left Vacuum Riemann problem at $t = 0.010$ for different Riemann solvers using Godunov's method
	}

\quickfigcap
	{./figures/godunov/GODUNOV-left_vacuum-1D-0.020.png}
	{fig:godunov-left-vacuum-3}
	{
		Left Vacuum Riemann problem at $t = 0.02$ for different Riemann solvers using Godunov's method
	}

\quickfigcap
	{./figures/godunov/GODUNOV-left_vacuum-1D-0.035.png}
	{fig:godunov-left-vacuum-4}
	{
		Left Vacuum Riemann problem at $t = 0.035$ for different Riemann solvers using Godunov's method
	}







%=======================================================================
\subsubsection{Vacuum Generating ICs}
%=======================================================================



\quickfigcap
	{./figures/godunov/GODUNOV-vacuum_generating-1D-0.001.png}
	{fig:godunov-vacuum-generating-1}
	{
		Vacuum Generating Riemann problem at $t = 0.001$ for different Riemann solvers using Godunov's method
	}
	


\quickfigcap
	{./figures/godunov/GODUNOV-vacuum_generating-1D-0.015.png}
	{fig:godunov-vacuum-generating-2}
	{
		Vacuum Generating Riemann problem at $t = 0.015$ for different Riemann solvers using Godunov's method
	}
	


\quickfigcap
	{./figures/godunov/GODUNOV-vacuum_generating-1D-0.030.png}
	{fig:godunov-vacuum-generating-3}
	{
		Vacuum Generating Riemann problem at $t = 0.030$ for different Riemann solvers using Godunov's method
	}
	

\quickfigcap
	{./figures/godunov/GODUNOV-vacuum_generating-1D-0.040.png}
	{fig:godunov-vacuum-generating-4}
	{
		Vacuum Generating Riemann problem at $t = 0.040$ for different Riemann solvers using Godunov's method
	}
	















%==============================================================
\subsection{Order of Convergence Study}
%==============================================================



\quickfigcap
	{./figures/godunov/godunov-convergence-dt-500.png}
	{fig:godunov-convergence-dt}
	{
		Testing the method's convergence with respect to the time step $\Delta t$ on Sod test initial conditions.
		For a fair comparison, both the cell width $\Delta x$ and the number of steps taken are fixed.
		The points are measurements, the lines are a linear fit, with the slope of the line given in the legend for each Riemann solver used in the legend.
	}


\quickfigcap
	{./figures/godunov/godunov-convergence-dx.png}
	{fig:godunov-convergence-dx}
	{
		Testing the method's convergence with respect to the cell width $\Delta x$ on Sod test initial conditions.
		For a fair comparison, the Courant number $C_{CFL}$ and the total number of steps are fixed.
		The points are measurements, the lines are a linear fit, with the slope of the line given in the legend for each Riemann solver used in the legend.
	}
	
\quickfigcap
	{./figures/godunov/godunov-convergence-CFL-10000.png}
	{fig:godunov-convergence-CFL}
	{
		Testing the method's convergence with respect to the Courant number $C_{CFL}$ on Sod test initial conditions.
		The points are measurements, the lines are just connecting them.
		The slope of $\half$ is plotted for comparison, and to demonstrate the deviation from it.
	}























%==============================================================
\subsection{Conclusions}
%==============================================================


\begin{itemize}


	\item 	Similar to the piecewise constant advection, the method is diffusive around sharp jump discontinuities.
			See figs. \ref{fig:godunov-1} - \ref{fig:godunov-4}.




	\item 	Godunov's method has some trouble dealing with vacuum.
			The issue is on one hand that the Riemann solver would characterize the vacuum state with some escape velocity, while if we compute the primitive values of a cell state in vacuum using the conserved variables, we have no way of predicting this escape velocity inside the cell;
			The cell might be in the interior of the vacuum state, so looking at neighbouring states and solving the Riemann problem there doesn't help.
			We'd have to keep looking to the left or to the right to find the appropriate neighbour state, and this would only be a solution for Riemann problems, not if we want to solve arbitrary problems.
			
			On the other hand, the code was reprogrammed to return zero velocity when the Riemann solvers are used in Godunov's scheme.
			This leads to very sharp jump discontinuities, see for example figs \ref{fig:godunov-left-vacuum-1} - \ref{fig:godunov-left-vacuum-4}.
			
			Similarly, some instabilities/inaccuracies/discontinuities develop in the vacuum-region in the vacuum generating case, figs. \ref{fig:godunov-vacuum-generating-1} - \ref{fig:godunov-vacuum-generating-4}.

	\item \textbf{Order of Convergence}
	
		\begin{itemize}
			
			\item 	Looking at the time step dependence (fig. \ref{fig:godunov-convergence-dt}), we always get slopes around $\sim 0.5$.
					Considering that a Sod test contains multiple jump discontinuities, this is absolutely as expected if we follow the same argumentation as for linear advection.
					See eq. \ref{eq:advection-error-square-root} and derivation leading up to it for comparison.
					All in all, it remains true that jump discontinuities reduce the order of convergence w.r.t. the time step.
					
			\item 	For the cell width dependence (fig. \ref{fig:godunov-convergence-dx}), we get a remarkable slope of $1.000$ for all solvers.
					Even more remarkable, all the $L1$ norms are identical.
					No approximate solver introduces more or less errors in this test case.
			
			\item 	For the $C_{cfl}$ dependence, we see that it deviates more stronger for higher $C_{CFL}$ from the $0.5$ power law that we get for small $\Delta t$.
					However, it is significantly better than what we get for advection (fig. \ref{fig:advection-convergence-CFL-step}).
					
					Why?
					
					Well, we don't necessarily have constant time steps any more, nor do we have constant velocities.
					It is conceivable that the errors ``correct themselves'' by demanding/allowing a smaller/larger timestep.
					
					What also might decrease the accuracy for high $C_{cfl}$ is that I  don't properly compute the emerging wave speeds, but estimate them;
					So there is a possibility that for high $C_{cfl}$, things are just computed wrongly, i.e. the chosen time step is too large to be stable or accurate.
			
		\end{itemize}
\end{itemize}