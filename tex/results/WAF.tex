%=============================================
\section{WAF Method}
%=============================================




%==============================================================
\subsection{Solving Riemann Problems using the WAF Method}
%==============================================================


\quickfigcap
	{./figures/WAF/WAF-1D-NO_LIMITER-riemann-sod-shock-1D.png}
	{fig:WAF-no-limiter}
	{
		Solution to the sod shock Riemann problem using the WAF method without flux limiters.
		The black line is the exact solution.
	}




\quickfigcap
	{./figures/WAF/WAF-sod_test_modified-EXACT-1D.png}
	{fig:WAF-riemann-const}
	{
		Solution to the modified sod test Riemann problem using the WAF method, an exact Riemann solver and various limiters.
		The black line is the exact solution.
	}


\quickfigcap
	{./figures/WAF/WAF-sod_test_modified-MC-1D.png}
	{fig:WAF-limiter-const-1}
	{
		Solution to the modified sod test Riemann problem using the WAF method, an MC limiter, and  various Riemann solvers.
		The black line is the exact solution.
	}

\quickfigcap
	{./figures/WAF/WAF-sod_test_modified-MINMOD-1D.png}
	{fig:WAF-limiter-const-2}
	{
		Solution to the modified sod test Riemann problem using the WAF method, a minmod limiter, and  various Riemann solvers.
		The black line is the exact solution.
	}

\quickfigcap
	{./figures/WAF/WAF-sod_test_modified-SUPERBEE-1D.png}
	{fig:WAF-limiter-const-3}
	{
		Solution to the modified sod test Riemann problem using the WAF method, a superbee limiter, and  various Riemann solvers.
		The black line is the exact solution.
	}

\quickfigcap
	{./figures/WAF/WAF-sod_test_modified-VAN_LEER-1D.png}
	{fig:WAF-limiter-const-4}
	{
		Solution to the modified sod test Riemann problem using the WAF method, a van Leer limiter, and various Riemann solvers.
		The black line is the exact solution.
	}






%=======================================================================
\subsection{Solving Vacuum Riemann Problems using WAF Method}
%=======================================================================


%=======================================================================
\subsubsection{Left Vacuum State}
%=======================================================================



\quickfigcap
	{./figures/WAF/WAF-left_vacuum-EXACT-1D.png}
	{fig:waf-left-vacuum-1}
	{
		Left Vacuum Riemann problem at for Exact Riemann solver using WAF method and various limiters
	}
	
\quickfigcap
	{./figures/WAF/WAF-left_vacuum-HLLC-1D.png}
	{fig:waf-left-vacuum-2}
	{
		Left Vacuum Riemann problem at  for HLLC Riemann solver using WAF method and various limiters
	}








%=======================================================================
\subsubsection{Vacuum Generating ICs}
%=======================================================================



\quickfigcap
	{./figures/WAF/WAF-vacuum_generating-EXACT-1D.png}
	{fig:waf-vacuum-generating-1}
	{
		Vacuum Generating Riemann problem with Exact Riemann solver using WAF method
	}
	


\quickfigcap
	{./figures/WAF/WAF-vacuum_generating-HLLC-1D.png}
	{fig:waf-vacuum-generating-2}
	{
		Vacuum Generating Riemann problem with HLLC Riemann solver using WAF method
	}













%==============================================================
\subsection{Order of Convergence Study}
%==============================================================



\quickfigcap
	{./figures/WAF/waf-convergence-dt-500.png}
	{fig:waf-convergence-dt}
	{
		Testing the method's convergence with respect to the time step $\Delta t$ on Sod test initial conditions.
		For a fair comparison, both the cell width $\Delta x$ and the number of steps taken are fixed.
		The points are measurements, the lines are a linear fit, with the slope of the line given in the legend for each Riemann solver used in the legend.
	}


\quickfigcap
	{./figures/WAF/waf-convergence-dx.png}
	{fig:waf-convergence-dx}
	{
		Testing the method's convergence with respect to the cell width $\Delta x$ on Sod test initial conditions.
		For a fair comparison, the Courant number $C_{CFL}$ and the total number of steps are fixed.
		The points are measurements, the lines are a linear fit, with the slope of the line given in the legend for each limiter used in the legend.
		The exact Riemann solver has been used.
	}
	
\quickfigcap
	{./figures/WAF/waf-convergence-CFL-10000.png}
	{fig:waf-convergence-CFL}
	{
		Testing the method's convergence with respect to the Courant number $C_{CFL}$ on Sod test initial conditions.
		The points are measurements, the lines are just connecting them.
		The slope of $\half$ is plotted for comparison, and to demonstrate the deviation from it.
		The exact Riemann solver has been used.
	}























%==============================================================
\subsection{Conclusions}
%==============================================================


\begin{itemize}

	\item 	The WAF method without limiters, fig. \ref{fig:WAF-no-limiter}, gives terrible results.
			In fact, in a lot of cases the oscillations would grow too strong and the code would crash.

	\item 	Similar to the piecewise linear advection, the drops around jump discontinuities are now much sharper, i.e. less diffusive, when limiters are applied.
			Compare fig. \ref{fig:WAF-riemann-const} to fig. \ref{fig:godunov-4}.

	\item 	The choice of the Riemann solver has no noticeable effect on the solution. 
			See fig. \ref{fig:WAF-riemann-const}.
	
	\item 	The effect of the limiter is comparable to how they behave on linear advection. 
			See fig. \ref{fig:WAF-limiter-const-1} - \ref{fig:WAF-limiter-const-4}.


	\item 	The trouble with vacuum that Godunov's method already had persists.
			I couldn't find literature on how to deal with vacuum using the WAF method, so I made up my own.
			For small time steps, it ``works'', as in it doesn't crash, and gives better results than the Godunov method if a left or a right state is vacuum.
			But for vacuum generating conditions, the velocity in the middle develops massive spikes earlier than Godunov's method;
			Compare figs. \ref{fig:waf-vacuum-generating-1} and \ref{fig:waf-vacuum-generating-2} to \ref{fig:godunov-vacuum-generating-2}.
			
			

	\item \textbf{Order of Convergence}
	
		\begin{itemize}
			
			\item 	Looking at the time step dependence (fig. \ref{fig:waf-convergence-dt}), it's absolutely terrible.
					I'm not sure why, but it is comparable to the results of WAF advection, fig. \ref{fig:advection-WAF-convergence-dt-step}. 
					
			\item 	For the cell width dependence (fig. \ref{fig:waf-convergence-dx}), we get a remarkable slope of $\approx 1$ for all solvers.

			
			\item 	For the $C_{cfl}$ dependence, it again doesn't follow the expected slope of the $0.5$ power law that we get for small $\Delta t$.
					Similarly to the WAF advection (fig. \ref{fig:advection-WAF-convergence-CFL-step}) however, it gets close to a power law (straight line in log space) already at ``high'' $C_{CFL} \approx 0.7$, which is telling us that the diffusion which is $\propto (1 - C_{CFL})$ for first order methods is strongly reduced.
		\end{itemize}
\end{itemize}