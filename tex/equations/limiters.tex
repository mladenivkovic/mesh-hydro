\newpage
\section{Slope and Flux Limiters}



\subsection{Slope Limiters}


Slope limiters are employed because issues arise around numerical schemes because of their discrete nature.
For example, a non-limited piecewise linear advection scheme will produce oscillations around jump discontinuities.
So the idea is to compute the slope in way that is useful for us based on the current situation of the gas state that we're solving for.

The choice of the slope can be expressed via a function $\phi(r)$ (see eqns. \ref{eq:advection_phi1}, \ref{eq:advection_phi2}) with 

\begin{align*}
	r^n_{i-\half} &= \begin{cases}
		\frac{\U_{i-1}^n - \U_{i-2}^n}{\U_{i}^n - \U_{i-1}^n} 	\quad \text{ for } \V  \geq 0 \\
		\frac{\U_{i+1}^n - \U_{i}^n}{\U_{i}^n - \U_{i-1}^n} 	\quad \text{ for } \V  \leq 0 \\
	\end{cases}
\end{align*}.


Possible limiters are:
\begin{flalign}
	\text{Minmod} 								&&\quad \phi(r) &= \mathrm{minmod}(1, r)\\
	\text{Superbee} 							&&\quad \phi(r) &= \max(0, \min(1, 2r), \min(2, r)) \\
	\text{MC (monotonized cenral-difference)} 	&&\quad \phi(r) &= \max(0, \min ((1+r)/2, 2, 2r))\\
	\text{van Leer}								&&\quad \phi(r) &= \frac{r + |r|}{1 + |r|}
\end{flalign}

where

\begin{align}
	\mathrm{minmod}(a, b) = 
		\begin{cases}
			a	& \quad \text{ if } |a| < |b| \text{ and } ab > 0\\
			b	& \quad \text{ if } |a| > |b| \text{ and } ab > 0\\
			0	& \quad \text{ if } ab \leq 0\\
		\end{cases}		
\end{align}