\newpage
%====================================================
\section{Advection}\label{chap:advection}
%====================================================



%====================================================
\subsection{Analytical Equation}
%====================================================

Advection is a bit of an exception as a hydrodynamics method because we're not actually solving the (ideal) gas equations, but these instead:

\begin{equation}
	\DELDT{ \U } + v \cdot \DELDX{\U} = 0
\end{equation}


Which is still a conservation law of the form

\begin{equation}
	\DELDT{ \U } + \DELDX{\F} = 0
\end{equation}

with the flux tensor 

\begin{equation}
	\F = v \cdot \U
\end{equation}

We assume the advection velocity $v = $ const.

Note that in the formalism used, we only solve the 1D advection, but for every component of the state vector $\U$ and flux tensor $\F$.

The analytical solution is given by any function $q(x)$ with $\U(x,t) = q(x - v t)$, which is just $q(x)$ translated by $v t$.


Why do we even bother with linear advection?
It's the simplest imaginable hyperbolic conservation law, and we can learn many things from considering this simple case.
Furthermore, a lot of mathematical background can't be (or at least isn't yet) proven for more complex conservation laws, so often the approach is to adapt the results of the linear advection to more complex situations, like for the Euler equations.







%====================================================
\subsection{Piecewise Constant Method}
%====================================================

\begin{figure}[htbp]
	\includegraphics[width=\textwidth]{./figures/piecewise_const.pdf}%
	\caption{Piecewise constant reconstruction of the field
		\label{fig:pwconst}
	}
\end{figure}


We assume that the cell state within a cell is constant (fig. \ref{fig:pwconst}).
Furthermore, we also assume that the velocity $v$ is constant and positive.


\begin{align}
	\U_i ^{n+1} &= 
		\U_i^{n} +  \frac{\Delta t}{\Delta x} \left( \F_{i-\half}^{n+\half} - \F_{i+\half}^{n+\half} \right) \label{eq:advection_basic}\\ 
	\F_{i\pm\half}^{n+\half} &= v_{i\pm\half}\cdot \U_{i - \half \pm \half} \label{eq:advection_flux}
\end{align}

The method is first order accurate in time and space.





We assumed that the velocity is positive and constant.
What if it's negative?

The important point is that we always do \textbf{upwind differencing}.
To obtain a finite difference, as we do here, you must never use the value that is downstream, i.e. that is in the direction of the flux.
Doing this means taking a value for your computation that won't be valid as soon as an infinitesimal time interval passes, because the ingoing flux will change the downwind state.
This is unphysical and leads to violent instabilities.

So if we have negative velocity, all we need to do is change the expression \ref{eq:advection_flux} to

\begin{equation}
	\F_{i\pm\half}^{n+\half} = v_{i\pm\half}\cdot \U_{i + \half \pm \half}
\end{equation}

















%====================================================
\subsection{Piecewise Linear Method}
%====================================================

\begin{figure}[htbp]
	\includegraphics[width=\textwidth]{./figures/piecewise_linear.pdf}%
	\caption{Piecewise linear reconstruction of the field
		\label{fig:pwlin}
	}
\end{figure}


The equation is still solved using the discretisation
\begin{align}
	\U_i ^{n+1} &= 
		\U_i^{n} +  \frac{\Delta t}{\Delta x} \left( \F_{i-\half}^{n+\half} - \F_{i+\half}^{n+\half} \right) \label{eq:advection_basic_pwlin}
\end{align}

but his time, we assume that the state is not constant within a cell, but follows a piecewise linear profile with some slope $\mathbf{s}$ (fig. \ref{fig:pwlin}):

\begin{align*}
	\text{For } x_{i-\half} < x < x_{i+\half}: &&
	& \U(x, t=t_n) = \U_i^n + \mathbf{s}_i^n(x - x_i) \\
	\text{Centered slope: (Fromm) } && &\mathbf{s_i}^n = \frac{\U_{i+1}^n - \U_{i-1}^n}{ 2 \Delta x}\\
	\text{Upwind slope: (Beam-Warming) } && &\mathbf{s_i}^n = \U_{i}^n - \U_{i-1}^n \\
	\text{Downwind slope: (Lax Wendroff) } && &\mathbf{s_i}^n = \U_{i+1}^n - \U_{i}^n \\
\end{align*}

The ``upwind'' and ``downwind'' slope namings are assuming that the velocity $v > 0$.
The implemented slope in the code is a centered slope.
The decision to go with a centered slope has no particular reason, I just felt like it.



Assuming a positive constant velocity $v$, we derive the flux $\F$ at the time $t^n < t < t^{n+1}$ at the interface position $i-\half$.
At time $t$, the cell will have been advected by a distance $v (t - t^n)$, and the current state at the interface will be

\begin{align*}
	\U(x=x_{i-\half}, t) 
		&= \U_{i-1}^n + \mathbf{s}_{i-1} (x_{i-\half} - v (t - t^n) - x_{i-1}) \\
		&= \U_{i-1}^n + \mathbf{s}_{i-1} (\frac{1}{2} \Delta x - v (t - t^n))
\end{align*}

To understand how the $x_{i-\half} - v (t - t^n)$ comes into play, imagine the state doesn't change (i.e. isn't advected), but you move the boundaries to the left instead over a distance $v(t - t^n)$.

So if we have a \textbf{negative} constant velocity, the term changes to 

\begin{align*}
	\U(x=x_{i-\half}, t) 
		&= \U_{i}^n + \mathbf{s}_{i} (x_{i-\half} - v (t - t^n) - x_{i}) \\
		&= \U_{i}^n + \mathbf{s}_{i} (-v (t - t^n) - \Delta x)
\end{align*}



Note that the minus sign remains, and that the indices changed by one because we need to always make sure to do upwind differencing, i.e. take only values where the flow comes from, not from the direction where it's going.

Finally, we can compute the average flux over the time step $\Delta t = t^{n+1} - t^{n}$:
\begin{align}
	\F_{i-\half}^{n+\half} 
	&= \langle \F_{i+\half}(t)\rangle _{t^n} ^ {t^{n+1}} 
	= \frac{1}{\Delta t} \int_{t^n}^{t^{n+1}} v \U(x 
	= x_{i-\half}, t) \\
	&= \frac{1}{\Delta t} \int_{t^n}^{t^{n+1}} v \left( \U_{i-1}^n + \mathbf{s}_{i-1} (\frac{1}{2} \Delta x - v (t - t^n)) \right) \\
	&= v \left( \U_{i-1}^n  + \mathbf{s}_{i-1} \left( \frac{1}{2} \Delta x - v \left( \left[ \frac{1}{2 \Delta t} t^2 \right]_{t^n}^{t^{n+1}} - t^n \right)  \right) \right) \\
	&= v \left( \U_{i-1}^n  + \mathbf{s}_{i-1} \left( \frac{1}{2} \Delta x - v \left[ \frac{1}{2} (t^{n+1} + t^n) - t^n \right] \right) \right) \\
	&= v \left( \U_{i-1}^n  + \frac{1}{2} \mathbf{s}_{i-1} \left(\Delta x - v \Delta t \right) \right)
\end{align}



Finally averaging the fluxes over a time step gives:
\begin{equation}
	\U_i^{n+1} = \U_i^n - v \cdot \frac{\Delta t}{\Delta x} ( \U_i ^n - \U_{i-1}^n) - v \cdot \frac{\Delta t}{\Delta x} \frac{1}{2} (\mathbf{s}_i^n - \mathbf{s}_{i-1}^n)(\Delta x - v \Delta t)
\end{equation}

This is the same as eq. \ref{eq:advection_basic} where we used
\begin{align*}
	\F_{i + \half}^{n+\half} 
		& = v_{i+\half} \cdot  \U_{i+\half} ^{n + \half} \\
		& = v \cdot \U(x_{i+\half} - \frac{1}{2} v \Delta t) \\
		& = v \cdot \left( \U_i^n + \mathbf{s}_i^n [(x_{i+\half} - \frac{1}{2} v \Delta t) - x_i ]  \right)\\
		& = v \cdot \left( \U_i^n + \frac{1}{2} \mathbf{s}_i^n (\Delta x -  v \Delta t)  \right)\\
\end{align*}

and analoguely
\begin{align*}
	\F_{i - \half}^{n+\half} 
		& = v \cdot \left( \U_{i-1}^n + \frac{1}{2} \mathbf{s}_{i-1}^n (\Delta x -  v \Delta t)  \right)\\
\end{align*}








To summarize the formulae:

\begin{align*}
	\U_i ^{n+1} &= 
		\U_i^{n} +  \frac{\Delta t}{\Delta x} \left( \F_{i-\half}^{n+\half} - \F_{i+\half}^{n+\half} \right) & \\
	\F_{i-\half}^{n+\half} &= 
		\begin{cases}
			v_{i-\half} \cdot \U_{i-1}^n +  \frac{1}{2} v_{i-\half} \cdot\mathbf{s}_{i-1}^n (\Delta x -  v_{i-\half} \Delta t)
			 	& \quad \text{for } v \geq 0 \\
			v_{i-\half} \cdot \U_{i}^n -\frac{1}{2} v_{i-\half} \cdot \mathbf{s}_{i}^n (\Delta x + v_{i-\half} \Delta t)
				& \quad \text{for } v \leq 0 \\
		\end{cases} \\
	\F_{i+\half}^{n+\half} &= 
		\begin{cases}
			v_{i+\half} \cdot \U_{i}^n +  \frac{1}{2} v_{i+\half} \cdot\mathbf{s}_{i}^n (\Delta x -  v_{i+\half} \Delta t)
			 	& \quad \text{for } v \geq 0 \\
			v_{i+\half} \cdot \U_{i+1}^n -\frac{1}{2} v_{i+\half} \cdot \mathbf{s}_{i+1}^n (\Delta x + v_{i+\half} \Delta t)
				& \quad \text{for } v \leq 0 \\
		\end{cases} \\		
\end{align*}





%We can now insert a more general expression for the slopes.
%Let 
%\begin{align}
%	\theta_{i-\half} = \begin{cases} +1 \quad \text{ for } v \geq 0 \\ -1 \quad  \text{ for } v \leq{0} \end{cases}
%\end{align}
%
%Then
%
%\begin{align}
%	\Delta x_{i-\{0, 1\}} \mathbf{s}_{i-\{0,1\}} 
%		&= \frac{1}{2} \Delta x \left[ (1 + \theta_{i-\half}) \mathbf{s}_{i-1}^n + (1 - \theta_{i-\half})  \mathbf{s}_{i}^n \right]  \\
%	&\equiv \phi(r_{i-\half}^n) (\U_i^n - \U_{i-1}^n ) \\
%	r^n_{i-\half} &= \begin{cases}
%		\frac{\U_{i-1}^n - \U_{i-2}^n}{\U_{i}^n - \U_{i-1}^n} 	\quad \text{ for } v  \geq 0 \\
%		\frac{\U_{i+1}^n - \U_{i}^n}{\U_{i}^n - \U_{i-1}^n} 	\quad \text{ for } v  \leq 0 \\
%	\end{cases} 
%\end{align}
%
%$\phi$ is discussed later. Finally:
%
%\begin{align}
%	\F_{i-\half}^{n+\half} = 
%		&\frac{1}{2} v_{i-\half} \left[  (1 + \theta_{i-\half}) \U_{i-1}^n + (1 - \theta_{i-\half})  \U_{i}^n \right] +\nonumber\\
%		&\frac{1}{2}| v_{i-\half} | \left( 1 - \left| \frac{v_{i-\half} \Delta t}{\Delta x} \right| \right) \phi(r_{i-\half}^n) (\U_i^n - \U_{i-1}^n ) \label{eq:advection_phi1}\\
%	\F_{i+\half}^{n+\half} = 
%		&\frac{1}{2} v_{i+\half} \left[  (1 + \theta_{i+\half}) \U_{i}^n + (1 - \theta_{i+\half})  \U_{i+1}^n \right] +\nonumber\\
%		&\frac{1}{2}| v_{i+\half} | \left( 1 - \left| \frac{v_{i+\half} \Delta t}{\Delta x} \right| \right) \phi(r_{i+\half}^n) (\U_{i+1}^n - \U_{i}^n ) \label{eq:advection_phi2}
%\end{align}
%





Let us introduce the unitless function $\phi(r)$ as

\begin{align}
	\phi(r_{i+\half}) &= \phi_{i+\half} = 
		\begin{cases}
			\frac{\Delta x}{\U_{i+1} - \U_{i}} \cdot \mathbf{s}_{i} 	& 	\text{ for } v \geq 0 \\
			\frac{\Delta x}{\U_{i+1} - \U_{i}} \cdot \mathbf{s}_{i+1}	&	\text{ for } v \leq 0 \\
		\end{cases}
		\\
	\phi(r_{i-\half}) &= \phi_{i-\half} = 
		\begin{cases}
			\frac{\Delta x}{\U_i - \U_{i-1}} \cdot \mathbf{s}_{i-1} 	& 	\text{ for } v \geq 0 \\
			\frac{\Delta x}{\U_i - \U_{i-1}} \cdot \mathbf{s}_{i}	&	\text{ for } v \leq 0 \\
		\end{cases}
\end{align}


Such that

\begin{align*}
	\F_{i-\half}^{n+\half} &= 
		\begin{cases}
			v_{i-\half} \cdot \U_{i-1}^n + \frac{1}{2} |v_{i-\half}|  \left(1 -  \frac{|v_{i-\half}| \Delta t}{\Delta x} \right) \phi_{i-\half} ( \U_i - \U_{i-1} )
			 	& \quad \text{for } v \geq 0 \\
			v_{i-\half} \cdot \U_{i}^n + \frac{1}{2} |v_{i-\half}|  \left(1 -  \frac{|v_{i-\half}| \Delta t}{\Delta x} \right) \phi_{i-\half} ( \U_i - \U_{i-1} )
				& \quad \text{for } v \leq 0 \\
		\end{cases} \\		
	\F_{i+\half}^{n+\half} &= 
		\begin{cases}
			v_{i+\half} \cdot \U_{i}^n + \frac{1}{2} |v_{i+\half}|  \left(1 -  \frac{|v_{i+\half}| \Delta t}{\Delta x} \right) \phi_{i+\half} ( \U_{i+1} - \U_{i} )
			 	& \quad \text{for } v \geq 0 \\
			v_{i+\half} \cdot \U_{i+1}^n + \frac{1}{2} |v_{i+\half}|  \left(1 -  \frac{|v_{i+\half}| \Delta t}{\Delta x} \right) \phi_{i+\half} ( \U_{i+1} - \U_{i} )
			 	& \quad \text{for } v \geq 0 \\
		\end{cases} \\		
\end{align*}






Depending on our choice of $\phi$, we can get different slopes. 
Here for positive velocity only, and for $r = r_{i-\half}$:

\begin{align*}
	\phi(r) & = 0 \rightarrow \mathbf{s}_i = 0 
		&\text{ No slopes; Piecewise constant method.}\\
	\phi(r) & = 1 \rightarrow \mathbf{s}_i = \frac{\U_{i} - \U_{i-1}}{\Delta x} 
		&\text{ Downwind slope (Lax-Wendroff)} \\
	\phi(r) & = r \rightarrow \mathbf{s}_i = \frac{\U_{i-1} - \U_{i-2}}{\Delta x} 
		&\text{Upwind slope (Beam-Warming)} \\
	\phi(r) & = \frac{1}{2} (1 + r) \rightarrow \mathbf{s}_i = \frac{\U_{i} - \U_{i-2}}{2 \Delta x} 
		&\text{ Centered slope (Fromm)} \\
\end{align*}


More elaborate expressions for $\phi(r)$ are going to be used to construct monotone high order schemes using a slope limiter, see section \ref{chap:limiters}.

















%====================================================
\subsection{Weighted Average Flux (WAF) Method}
%====================================================



For the WAF method, we again assume piece-wise constant data (see fig. \ref{fig:pwconst}), i.e.

\begin{equation}
	\U_i ^ n = \frac{1}{\Delta \x} \int_{\x_{i-\half}}^{\x_{i+\half}} \U(\x, t^n) \de \x
\end{equation}

The scheme is again based on the explicit conservative formula

\begin{equation}
	\U_{i}^{n+1} = \U_i^n + \frac{\Delta t}{\Delta x} \left[ \F_{i - \half} - \F_{i + \half} \right] \label{eq:advection_basics_waf}
\end{equation}



The intercell flux $\F_{i + \half}$ is defined as an integral average of the flux function:

\begin{equation}
	\F_{i + \half} = \frac{1}{\Delta x} \int_{-\frac{1}{2} \Delta x} ^{\frac{1}{2} \Delta x} \F (\U_{i+\half} ( x, \frac{1}{2}\Delta t)) \de x \label{eq:waf-flux}
\end{equation}

The integration range goes from the middle of the cell to the middle of the neighbouring cell.

The solution for $\U_{i+\half} ( x, \frac{1}{2}\Delta t)$ for linear advection is simple; it is

\begin{align}
	\U_{i+\half} ( x, t) = 
		\begin{cases}
			\U_i 		& \frac{x}{t} < v \\
			\U_{i+1} 	& \frac{x}{t} > v \\
		\end{cases}
\end{align}



\begin{figure}[htbp]
	\includegraphics[width=\textwidth]{./figures/WAF.pdf}%
	\caption{Figure to show the derivation of the WAF intercell flux (eq. \ref{eq:waf-flux}) for linear advection.
		We have two piecewise constant states, $\U_i$ and $\U_{i+1}$, separated at the position $x_{i+\half}$.
		For $v > 0$, the state, and hence the flux, is constant over the intervals $\overline{AB^+}$ and $\overline{B^+C}$.
		For $v < 0$, it is constant over the intervals $\overline{AB^-}$ and $\overline{B^-C}$.
		\label{fig:waf}
	}
\end{figure}



and the evaluation of the flux integral is trivial since the solution consists of constant states:
See figure \ref{fig:waf}.
For $v > 0$, the state, and hence the flux, is constant over the intervals $\overline{AB^+}$ and $\overline{B^+C}$;
For $v < 0$, the state, and hence the flux, is constant over the intervals $\overline{AB^-}$ and $\overline{B^-C}$.

It's easy to show that
\begin{align*}
	\overline{AB^+} &= \frac{1}{2} \Delta x + v \cdot \frac{1}{2} \Delta t\\
	\overline{B^+C} &= \frac{1}{2} \Delta x - v \cdot \frac{1}{2} \Delta t\\
	\overline{AB^-} &= \frac{1}{2} \Delta x - |v| \cdot \frac{1}{2} \Delta t\\
	\overline{B^-C} &= \frac{1}{2} \Delta x + |v| \cdot \frac{1}{2} \Delta t\\
\end{align*}









So in either case for $v > 0$ and $v < 0$ we obtain the final expression for the flux

\begin{equation}
	\F_{i+\half} =  
		\frac{1}{2} (1 + c) v \U_i^n + \frac{1}{2} (1 - c) v \U_{i+1}
\end{equation}

where

\begin{equation}
	c = \frac{v \Delta t}{\Delta x}
\end{equation}

and is allowed to be negative.


This gives us a second order accurate method despite having piecewise constant data.

Second order methods will have spurious oscillations near discontinuities and steep gradients, which we try to handle with flux limiters.
To this end, a  flux limited WAF flux can be written as

\begin{equation}
	\F_{i+\half}^{n+\half} =  
		\frac{1}{2} (1 + sign(v) \psi_{i+\half})\ v \U_i^n + \frac{1}{2} (1 - sign(v) \psi_{i+\half})\ v \U_{i+1}^n
\end{equation}

possible choices for flux limiters $\psi$ are discussed in section \ref{chap:limiters}. 
The choice
\begin{align*}
	\psi = |c|
\end{align*}
recovers the original expression.









%====================================================
\subsection{CFL Condition}
%====================================================

To keep things stable and physical, we must not allow any flux in the simulation to go further than one single cell size.
Otherwise, you're skipping interactions between fluxes on cells.
This time restriction is known as the CFL condition.

In 1D, it's straightforward:

\begin{equation}
	\Delta t_{max} = C_{cfl} \frac{\Delta x}{v_{max}} \label{eq:CFL1D}
\end{equation}

$C_{cfl} \in [0, 1) $ is a user-set factor.
The lower it is, the more precise the results, but the more computations you need to do.

In 2D, it is:
\begin{equation}
	\Delta t_{max} = C_{cfl} \left( \frac{|v_{x,max}|}{\Delta x} +  \frac{|v_{y,max}|}{\Delta y} \right)^{-1} \label{eq:CFL2D}
\end{equation}

This condition is more strict than what one would expect from the restriction based on physical arguments, i.e. not allowing the flux to pass more than one cell, which would be $\Delta t_{max} = C_{cfl} \min \left\{ \frac{\Delta x}{|v_{x,max}|} ,  \frac{\Delta y}{|v_{y,max}|} \right\} $.
It follows from a convergence condition in (von Neumann) stability analysis of the method.

For $N$ dimensions, the condition translates to

\begin{equation}
	\Delta t_{max} = C_{cfl} \left( \sum_{i=1}^{N} \frac{|v_{i,max}|}{\Delta x_i} \right)^{-1}  \label{eq:CFLND}
\end{equation}






















%====================================================
\subsection{Implementation Details}
%====================================================






%====================================================
\subsubsection{General}
%====================================================


What is implemented is the equation

\begin{equation}
    \DELDT{\U} + v \cdot \DELDX{\U} = 0
\end{equation}


where we assume that the velocity $v$ is constant.

In every case, this equation is solved through the discretization

\begin{equation}
	\U_i^{n+1} = \U_i^n + \frac{\Delta t}{\Delta x} \left( \F_{i - \half}^{n+\half} - \F_{i + \half}^{n+\half} \right) \label{eq:advection-discretized}
\end{equation}

and the expressions for the fluxes $\F_{i\pm\half}$ vary between the methods.
 
Therefore, the fluid velocity is never updated, but kept identical to the initial conditions.
You can change that behaviour by removing the \verb|ADVECTION_KEEP_VELOCITY_CONSTANT| macro definition in \verb|defines.h|

Even though usually $\U$ is taken to be a conserved state, advection only updates the primitive states, as those are the one that are read in and written out.



The advection related functions are called in the main loop in \texttt{/program/src/main.c} when \verb|solver_step(...)| is called.
The specifics for each solver are described in the following subsection.







%====================================================
\subsubsection{Piecewise Constant Advection}
%====================================================


The related functions are written in \verb|/program/src/solver/advection_pwconst.c| and \verb|/program/src/solver/advection_pwconst.h|.

The \verb|solver_step(...)| function does the following for the 1D case:
\begin{itemize}
	\item 	Reset the stored fluxes from the previous timestep to zero
	\item 	Impose boundary conditions (section \ref{chap:boundary-conditions})
	\item 	Find the maximal timestep that you can do by applying the CFL condition \ref{eq:CFL1D}.
	\item 	Compute fluxes:
	\begin{itemize}
		\item 	For every cell, find the upwind and the downwind cell for the interaction.
		\item 	Compute the net flux, i.e.
				\begin{equation}
				\F_{i, net}^{n+\half} = \F_{i-\half}^{n+\half} - \F_{i+\half}^{n+\half}
				\end{equation}
				and store it in the \texttt{struct pstate pflux} struct of the cell $i$.
				\texttt{struct pstate} is a struct that contains the primitive state, i.e. density $\rho$, velocity $u_x$, $u_y$, and pressure $p$.
	\end{itemize}
	\item 	Update the states: Effectively compute $\U^{n+1}$ at this point using $\U_i^n$ and the computed net flux $\F_{i, net}^{n+\half}$ according to eqn. \ref{eq:advection-discretized}.
\end{itemize}



For 2D, we just do the 1D step twice, once in each direction while using the intermediate result coming from the first sweep as the IC for the second step in the other dimension.
See section \ref{chap:dimensional-splitting} for details.









%====================================================
\subsubsection{Piecewise Linear Advection}
%====================================================

The related functions are written in \verb|/program/src/solver/advection_pwlin.c| and \verb|/program/src/solver/advection_pwlin.h|.


The \verb|solver_step(...)| function does the following for the 1D case:
\begin{itemize}
	\item 	Reset the stored fluxes from the previous timestep to zero
	\item 	Impose boundary conditions (section \ref{chap:boundary-conditions})
	\item 	Find the maximal timestep that you can do by applying the CFL condition \ref{eq:CFL1D}.
	\item 	Compute fluxes:
	\begin{itemize}
		\item 	For every cell, find the upwind and the downwind cell for the interaction.
		\item 	Compute the net flux, i.e.
				\begin{equation}
				\F_{i, net}^{n+\half} = \F_{i-\half}^{n+\half} - \F_{i+\half}^{n+\half}
				\end{equation}
				and store it in the \texttt{struct pstate pflux} struct of the cell $i$.
				\texttt{struct pstate} is a struct that contains the primitive state, i.e. density $\rho$, velocity $u_x$, $u_y$, and pressure $p$.
		\item 	When computing the fluxes $\F_{i-\half}^{n+\half}$ and $\F_{i+\half}^{n+\half}$, first just compute the left and right slope by calling the \texttt{limiter_get_slope_left(...)} and \texttt{limiter_get_slope_right(...)} functions, and only determine later which one is the upwind one and which one is the downwind one.
				The \verb|limiter_get_slope| functions will figure out which limiter to use, even if you chose the ``no limiter'' option.
				If the ``no limiter'' option is chosen, you get a centered slope.
	\end{itemize}
	\item 	Update the states: Effectively compute $\U^{n+1}$ at this point using $\U_i^n$ and the computed net flux $\F_{i, net}^{n+\half}$ according to eqn. \ref{eq:advection-discretized}.
\end{itemize}



For 2D, we just do the 1D step twice, once in each direction while using the intermediate result coming from the first sweep as the IC for the second step in the other dimension.
See section \ref{chap:dimensional-splitting} for details.










%====================================================
\subsubsection{WAF Advection}
%====================================================


The \verb|solver_step(...)| function does the following for the 1D case:
\begin{itemize}
	\item 	Reset the stored fluxes from the previous timestep to zero
	\item 	Impose boundary conditions (section \ref{chap:boundary-conditions})
	\item 	Find the maximal timestep that you can do by applying the CFL condition \ref{eq:CFL1D}.
	\item 	Compute fluxes:
	\begin{itemize}
		\item 	For every cell, find the pass it and it's neighbour with index $i+1$ to compute the flux.
		\item 	Compute the flux, i.e. $ \F_{i+\half}^{n+\half} $, and subtract it from the \texttt{struct pstate pflux} struct of the cell $i$.
				\texttt{struct pstate} is a struct that contains the primitive state, i.e. density $\rho$, velocity $u_x$, $u_y$, and pressure $p$.
				But also add it to the cell with index $i+1$, since $\F_{i+\half}$ will the the flux $\F_{i-\half}$ for the cell with $i = i+1$.
				This way, once the iteration is over, the net flux $\F_{i-\half}^{n+\half}$ and $\F_{i+\half}^{n+\half}$ will be stored in \texttt{struct pstate pflux} of the cell $i$.
	\end{itemize}
	\item 	Update the states: Effectively compute $\U^{n+1}$ at this point using $\U_i^n$ and the computed net flux $\F_{i, net}^{n+\half}$ according to eqn. \ref{eq:advection-discretized}.
\end{itemize}


For 2D, we just do the 1D step twice, once in each direction while using the intermediate result coming from the first sweep as the IC for the second step in the other dimension.
See section \ref{chap:dimensional-splitting} for details.

