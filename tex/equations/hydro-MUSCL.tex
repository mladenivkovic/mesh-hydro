%=====================================================================
\subsection{The MUSCL-Hancock Method} \label{chap:MUSCL-hancock}
%=====================================================================











%======================================================
\subsubsection{Method}
%======================================================

The MUSCL-Hackock method is a method of the class of \textbf{M}onotone \textbf{U}pwind \textbf{S}chemes for \textbf{C}onservation \textbf{L}aws, which try to achieve higher order accuracy by describing the states $\U_i$ of each cell $i$ not by a constant state, but by some higher order interpolation.
The MUSCL-Hanckock scheme in particular assumes a piecewise linear state reconstruction, see fig. \ref{fig:piecewise-linear} for an example.


\begin{figure}[H]
	\includegraphics[width=\textwidth]{./figures/piecewise_linear.pdf}%
	\caption{	
		A piecewise linear representation of continuous data among cells.
		\label{fig:piecewise-linear}
		}
\end{figure}


It solves the hyperbolic conservation law of the form

\begin{equation*}
	\DELDT{\U} + \DELDX{\F(\U))} = 0
\end{equation*}

just like all other schemes so far using the explicit conservative formula

\begin{equation}
	\U_i^{n+1} = \U_i^n + \frac{\Delta t}{\Delta x} \left( \F_{i-\half} - \F_{i + \half} \right)
\end{equation}

where $\U_i$ is the volume average of a state in a cell, which is however not constant throughout the cell, but is described by

\begin{equation}
	\U_{i}(x) = \U_i^n + \frac{x - x_i}{\Delta x} \mathbf{s}_i; \quad x \in [0, \Delta x]
\end{equation}


$\mathbf{s}_i$ is a suitably chosen slope vector of $\U_i(x)$ in cell $i$.
A general slope can be written as

\begin{equation}
 	\mathbf{s}_i = \frac{1}{2} (1 + \omega) (\U_i - \U_{i-1}) + \frac{1}{2} (1 - \omega) (\U_{i+1} - \U_i)
\end{equation}

with $\omega \in [-1, 1]$.
For $\omega = 0$, we retrieve the centered scheme.
$\omega = 1$ gives us the upwind slope, $\omega = -1$ gives us the downwind slope.


However, having non-constant states creates a problem for Riemann solvers.
We now need to solve the so called \emph{Generalised Riemann Problem} with

\begin{align}
	\DELDT{\U} + \DELDX{\F(\U))} &= 0 \\
	\U(x, 0) &= 
	\begin{cases}
		\U_i(x), & \quad x < 0 \\
		\U_{i+1}(x), & \quad x > 0 \\
	\end{cases}
\end{align}

As the left and right states change with $x$, the characteristics are no longer straight lines, which makes things tricky.
So instead of dealing with that analytically, the MUSCL-Hancock method tries to compute some sort of ``intermediate state'' such that in the end, we get an approximate flux that is good enough for our purposes.

In each cell, the extreme values are located at the cell boundaries.
They are referred to as boundary extrapolated values, and are given by

\begin{equation}
	\U_i^L = \U_i^n - \frac{1}{2} \mathbf{s}_i; \quad 	\U_i^R = \U_i^n + \frac{1}{2} \mathbf{s}_i; 
\end{equation}


To obtain a second order intercell flux, we first try and find intermediate extrapolated boundary values by applying the same conservative explicit formula on every cell separately over half the timestep.
We denote the intermediate boundary extrapolated values as $\overline{\U}_i^L$ and $\overline{\U}_i^R$, and obtain them by computing

\begin{align}
	\overline{\U}_i^L &= \U_i^L + \frac{1}{2} \frac{\Delta t}{\Delta x} \left( \F(\U_i^L) - \F(\U_i^R) \right)\\
	\overline{\U}_i^R &= \U_i^R + \frac{1}{2} \frac{\Delta t}{\Delta x} \left( \F(\U_i^L) - \F(\U_i^R) \right)
\end{align}

Note that this update depends only on the values inside a cell, and can be computed for every cell individually.

Finally, with the evolved boundary extrapolated values, we can compute the fluxes $\F_{i + \half} = \F(\U_{i+\half}(x = 0))$ by solving the Riemann problem at every cell interface and using the initial values

\begin{align}
	\U_L &= \overline{\U}_i^R\\
	\U_R &= \overline{\U}_{i+1}^L 
\end{align}

and by sampling the solution at $x = 0$.


A TVD version is obtained by using slope limiters, i.e. replacing the slope $\mathbf{s}_i$ by a limited slope $\overline{\mathbf{s}}_i$ with

\begin{align*}
	\overline{\mathbf{s}}_i &= \xi(r) \mathbf{s}_i \\
	r &= \frac{\U_i - \U_{i-1}}{\U_{i+1} - \U_{i}} \quad\quad \text{for each component of }\U
\end{align*} 

some possible and implemented slope limiters $\xi(r)$ are given in section \ref{chap:implemented_limiters}.







%======================================================
\subsubsection{Implementation Details}
%======================================================