\newpage

%==============================================
\section{Hydrodynamics Methods}\label{chap:hydro}
%==============================================


%==============================================
\subsection{Godunov's Method}
%==============================================










%============================================
\subsubsection{Method}
%============================================

Godunov's method arises from the integral form of the conservation law so that discontinuous solutions are allowable.

For the one dimensional method, we discretize the spatial domain into $M$ computing cells of regular size, and assume that the initially continuous data is represented by piecewise constant distribution of data, see fig. \ref{fig:piecewise-constant}.



\begin{figure}[H]
	\includegraphics[width=\textwidth]{./figures/piecewise_const.pdf}%
	\caption{	
		A piecewise constant representation of continuous data among cells.
		\label{fig:piecewise-constant}
		}
\end{figure}

Having a collection of piecewise constant states, we effectively have to solve local Riemann problems with data $\U_i$ as $\U_L$ and $\U_{i+1}$ as $\U_R$, centered at the intercell boundary positions $x_{i+\half}$.
The solution of the Riemann problem will depend on $\frac{\bar{x}}{\bar{t}}$, where $\bar{x}$ and $\bar{t}$ are in local coordinates to the specific Riemann problem under consideration. 
$\bar{x}$ is zero at $x_{i+\half}$ and increasingly negative with decreasing $i$.
$\bar{t}$ is zero at the current timestep.


Now suppose that we have solved the Riemann problem at the position $x_{i+\half}$ with left state $\U_L = \U_i$ and right state $\U_R = \U_{i+1}$.
Then, as time evolves, how will the state at $x_{i+\half}$ change?

Recall that the elementary waves travel along characteristics, and that the characteristics are straight lines on the $x-t$ - diagram (see fig. \ref{fig:riemann-solution}.
Then the state at $x_{i+\half}$, which is where the dividing line between the two initial states is, will be given by the solution of the Riemann problem at the position $\bar{x} = 0$, and will remain the same for all $\bar{t} > 0$.
(This is assuming there is nothing else that might disturb the current situation.)


Using that fact, it can be derived that (see \cite{toro}) 

\begin{align}
	\U^{n+1}_i = \U^n_i + \frac{\Delta t}{\Delta x} \left[\F(\U_{i-\half}) - \F(\U_{i+\half}) \right] \label{eq:godunov-discretized}
\end{align}

where $\U_{i-\half}$ and $\U_{i+\half}$ are the solutions to the Riemann problems at $x_{i-\half}$ and $x_{i+\half}$, respectively.


It is noteworthy that this is an exact solution to a piecewise constant initial state.




Lastly, we need to limit the time step size.
We mustn't allow for a wave to be able to travel further than one cell length between two timesteps, otherwise we get bogus results.
Remember that we assumed that the state at $x_{i+\half}$ doesn't change after $\bar{t} > 0$.
This is only satisfied if the wave doesn't reach the boundary of the neighbouring cell.

This time step restriction is imposed by the CFL condition:

\begin{align}
	\Delta t_{max} \leq \frac{C_{cfl} \Delta x}{|S_{max}^n|} \label{eq:godunov-cfl}
\end{align}


where $S_{max}^n$ is the highest wave propagation speed at the current time, and $C_{cfl} \in [0, 1($ is the Courant number.


However, this is not the most practical way of doing things.
In 2D, using dimensional splitting, we'd have to first solve everything in one direction to find the wave speeds, then advance the time step of the sweep, then do the other sweep and hope that the maximal wave velocity won't be greater than the one of the previous sweep.
Or re-do the first sweep iteratively until we get decent time steps.

Instead, we use the estimate

\begin{align}
	S^n_{max} = \max\{ |u_i^n| + a_i^n \}
\end{align}

This is not always accurate, and the wave speed can be underestimated, leading to instabilities.
To combat this, we just need to choose a lower $C_{cfl}$.
Toro recommends to use $C_{cfl} < 0.8 - 0.9$.













%===================================================================
\subsubsection{Implementation Details}\label{chap:godunov-details}
%===================================================================


The cells are stored as an array of \verb|struct cell| that stores both primitive states, \texttt{prim}, as  \texttt{struct pstate}, and conserved states, \texttt{cons}, as \texttt{struct cstate}.
Furthermore they have \texttt{struct pstate pflux} and \texttt{struct cstate cflux} to store fluxes of primitive and conserved variables.


The grid is set up as follows:
In 1D, it is a 1D array of \texttt{struct cell}.
In 2D, it is a 2D array.
Indices (0, 0) represent the lower left corner of the simulation domain.
First index is in x direction, i.e. (nx - 1, 0) is at the coordinates (x = xmax, y = 0).


The flux at $x_{i+\half, j}$ and $y_{i, j+\half}$ are stored in \texttt{cflux} or \texttt{pflux} of cell \verb|grid[i, j]|, depending whether you're storing primitive or conserved variables.
For the Godunov scheme, we need conserved variables.
For advection, we only deal with primitive variables.
Because we're doing dimensional splitting, it suffices to have only one storage place, as they will be used in successive order.
See section \ref{chap:dimensional-splitting} for details.

We can afford to store $x_{i+\half}$ at cell $i$ because we have at least 1 extra virtual boundary cell which is used to apply boundary conditions, so the flux at $x_{-\half}$ will be stored in \verb|grid[BC-1]|, where \texttt{BC} is the number of boundary cells used, defined in \texttt{defines.h}.
 

If the grid is only in 1D, then all the above definitions still apply as if y didn't exist.



The related functions are written in \texttt{/program/src/solver/godunov.c} and \texttt{/program/src/solver/godunov.h}.
The hydro related functions are called in the main loop in \texttt{/program/src/main.c} when \verb|solver_step(...)| is called.

The \verb|solver_step(...)| function does the following for the 1D case:
\begin{itemize}
	\item 	Reset the stored fluxes from the previous timestep to zero
	\item 	Compute the primitive states for all cells from the updated conserved states
	\item 	Impose boundary conditions (section \ref{chap:boundary-conditions})
	\item 	Find the maximal timestep that you can do by applying the CFL condition \ref{eq:godunov-cfl}.
	\item 	Compute fluxes:
	\begin{itemize}
		\item 	For every cell pair $(i, i+1)$, solve the Riemann problem (see section \ref{chap:riemann}) to find the flux $\F_{i+\half}$.
		\item 	Store the flux $\F_{i+\half}$ in the \texttt{struct pstate pflux} struct of the cell $i$.
				\texttt{struct pstate} is a struct that contains the primitive state, i.e. density $\rho$, velocity $u_x$, $u_y$, and pressure $p$.
	\end{itemize}
	\item 	Update the states: Effectively compute $\U^{n+1}$ at this point using $\U_i^n$, the flux $\F_{i+\half}$ stored in every cell $i$, and the flux $\F_{i-\half}$ stored in every cell $i-1$ following eq. \ref{eq:godunov-discretized}.
\end{itemize}

























%==============================================
\subsection{Weighted Average Flux (WAF) Method}
%==============================================







%============================================
\subsubsection{Method}
%============================================


For the WAF method, we again assume piece-wise constant data (see fig. \ref{fig:piecewise-constant}), i.e.

\begin{equation}
	\U_i ^ n = \frac{1}{\Delta \x} \int_{\x_{i-\half}}^{\x_{i+\half}} \U(\x, t^n) \de \x
\end{equation}

The scheme is again based on the explicit conservative formula

\begin{equation}
	\U_{i}^{n+1} = \U_i^n + \frac{\Delta t}{\Delta x} \left[ \F_{i - \half} - \F_{i + \half} \right] \label{eq:hydro_basics_waf}
\end{equation}



The intercell flux $\F_{i + \half}$ is defined as an integral average of the flux function:

\begin{equation}
	\F_{i + \half} = \frac{1}{\Delta x} \int_{-\frac{1}{2} \Delta x} ^{\frac{1}{2} \Delta x} \F (\U_{i+\half} ( x, \frac{1}{2}\Delta t)) \de x \label{eq:hydro-waf-flux}
\end{equation}

The integration range goes from the middle of the cell to the middle of the neighbouring cell.





\begin{figure}[htbp]
	\includegraphics[width=\textwidth]{./figures/WAF-hydro.pdf}%
	\caption{Figure to show the derivation of the WAF intercell flux (eq. \ref{eq:hydro-waf-flux}) for the 1D Euler equations.
		We have initially two piecewise constant states, $\U_i$ and $\U_{i+1}$, separated at the position $x_{i+\half}$.
		As time evolves, three waves will emerge, with respective speeds $S_1$, $S_2$, and $S_3$.
		The states between the points $A_k$, $A_{k+1}$ are assumed constant.
		\label{fig:hydro-waf}
	}
\end{figure}



The solution of the Riemann problem separates the two initial states $\U_L = \U_i$, $U_R = \U_{i+1}$ into four states

\begin{align*}
	\U^{(1)} = \U_L, \quad	\U^{(2)} = \U_L^*, \quad	\U^{(3)} = \U_R^*, \quad	\U^{(4)} = \U_R
\end{align*}

that are separated by three waves with the speeds $S_1$, $S_2$, and $S_3$.
At $t = \frac{1}{2} \Delta t$, we can separate the interval $[-\Delta x /2, \Delta x /2]$ by introducing 5 points along the $x$ axis:

\begin{align*}
	A_0 &= - \Delta x / 2\\
	A_1 &= S_1 \Delta t / 2\\
	A_2 &= S_2 \Delta t / 2\\
	A_3 &= S_3 \Delta t / 2\\
	A_4 &= \Delta x / 2
\end{align*}

and separate the integral \ref{eq:hydro-waf-flux} into the sum
\begin{align}
\F_{i + \half} = \frac{1}{\Delta x} \sum\limits_{k = 1}^{N+1} \int\limits_{A_{k-1}}^{A_k} \F (\U(x, \Delta t / 2)) \de x
\end{align}

where $N$ is the number of occuring waves in the solution.





Note that with $|C_{cfl}| \leq 1$ we should always have all the waves within $[-\Delta x /2, \Delta x / 2]$ at $t = \Delta t / 2$ if the $C_{cfl}$ is chosen properly using actual wave speeds.
However, we use approximate wave speed estimates, so we need to check whether the actual waves are still inside $[-\Delta x /2, \Delta x / 2]$ at $t = \Delta t / 2$.



Since we assume constant states between these points (which they will be unless we have a rarefaction present), the fluxes $\F$ between the points $A_k$ will be constant too, and the integral is trivial.
We only need expressions for the distances $\overline{A_{k}A_{k+1}}$.
It's easy to show that regardless of the sign of the wave speeds $S_k$, we obtain

\begin{align*}
	\overline{A_0 A_1} &= 
		\frac{\Delta x}{2} ( 1 + c_1 ) \\
	\overline{A_1 A_2} &= 
		\frac{\Delta x}{2} ( c_2 - c_1 ) \\
	\overline{A_2 A_3} &= 
		\frac{\Delta x}{2} ( c_3 - c_2 ) \\
	\overline{A_3 A_4} &= 
		\frac{\Delta x}{2} ( 1 - c_3 ) \\ 
	\text{ with } c_k &= \frac{S_k \Delta t}{\Delta x}
\end{align*}



If we define

\begin{align*}
	\beta_k &= \frac{\overline{A_{k-1} A_k}}{\Delta x}\\
	c_0 &= -1, \quad c_5 = c_{N+1} = 1
\end{align*}

we obtain

\begin{align*}
	\beta_k &= \frac{1}{2} (c_k - c_{k-1})
\end{align*}

and we can write the WAF flux as

\begin{align}
	\F_{i + \half} 
		&= \sum\limits_{k = 1}^{N+1} \beta_k \F^{(k)} \\
		&= \frac{1}{2} (\F_i + \F_{i+1}) - \frac{1}{2} \sum\limits_{k = 1}^{N} c_k \left (\F^{(k+1)} - \F^{(k)} \right) 
\end{align}




The TVD modification of the WAF flux is


\begin{align}
	\F_{i + \half} 
		&= \frac{1}{2} (\F_i + \F_{i+1}) - \frac{1}{2} \sum\limits_{k = 1}^{N} sign(c_k) \psi_{i+\half}^{(k)} \left (\F^{(k+1)} - \F^{(k)} \right) \\
	\psi_{i+\half}^{(k)}
		&= \psi_{i+\half}(r^{(k)}) \\
	r^{(k)} &=
		\begin{cases}
			\frac{\Delta q_{i-\half}^{(k)}}{\Delta q_{i+\half}^{(k)}}	& \text{ if } c_k > 0 \\[2em]
			\frac{\Delta q_{i+3/2}^{(k)}}{\Delta q_{i+\half}^{(k)}}	& \text{ if } c_k < 0 \\		
		\end{cases}
\end{align}

Where $q$ is one single quantity which is known to change across every wave.
Options are density $\rho$ and internal energy $\epsilon$.
I implemented the choice $\rho$.

The limiters $\psi$ are related to conventional flux limiters $\phi$ via

\begin{equation}
	\psi_{i+\half} = 1 - ( 1 - |c|) \phi_{i+\half}(r)
\end{equation}

Some implemented options are given in section \ref{chap:implemented_limiters}.















%===================================================================
\subsubsection{Implementation Details}\label{chap:hydro-WAF-details}
%===================================================================


