\newpage
\section{Solving Multidimensional Problems: Dimensional Splitting}\label{chap:dimensional-splitting}


To go from one to multiple dimensions, it is tempting to just extend the one dimensional discretisation.
For example, starting with the conservation law

\begin{equation}
	\deldt{\U} + \deldx{\F(\U)} + \frac{\del}{\del y} \mathbf{G}(\U) = 0
\end{equation}


and just apply Godunov's finite volume method:

\begin{equation}
	\U_{i,j}^{n+1} = 
					\U_{i,j}^n 
					+ \frac{\Delta t}{\Delta x} \left( \F_{i - \half, j} - \F_{i + \half, j}\right) 
					+ \frac{\Delta t}{\Delta x} \left( \mathbf{G}_{i ,j - \half} - \mathbf{G}_{i, j + \half}\right) 
\end{equation}


However, this is a bit of a problem.
The upwinding here is not complete.
Consider the 2D advection equation

\begin{equation}
	\deldt{q} + u\deldx{q} + v \frac{\del}{\del y} q = 0
\end{equation}

Now suppose we have advecting velocities $u = v = 1$, i.e. the advection velocity is along the diagonal.
Then our method reads 

\begin{equation}
	q_{i,j}^{n+1} = 
					q_{i,j}^n 
					+ u \frac{\Delta t}{\Delta x} \left( q_{i - \half, j} - q_{i + \half, j}\right) 
					+ v \frac{\Delta t}{\Delta x} \left( q_{i ,j - \half} - q_{i, j + \half}\right) 
\end{equation}

This expression doesn't involve $q_{i-1, j-1}$ at all, but that's actually the value that should be advected to $q_{i, j}$ in the next timestep!


One way of doing things is to actually formulate more sophisticated unsplit methods that involve the appropriate stencils.
We shan't do that here though.
We make use of dimensional splitting. 
Instead of solving

\begin{equation}
	\begin{cases}
		\text{PDE: } & \deldt{\U} + \deldx{\F(\U)} + \frac{\del}{\del y} \mathbf{G}(\U) = 0\\
		\text{IC: } & \U(x, y, t^n) = \U^n
	\end{cases}
\end{equation}

we do it in 2 (3 for 3D) steps:

Step 1: We obtain a intermediate result $\U^{n+\half}$ by solving the ``x - sweep'' over the full time interval $\Delta t$ :
\begin{equation}
	\begin{cases}
		\text{PDE: } & \deldt{\U} + \deldx{\F(\U)} = 0\\
		\text{IC: } &  \U^n
	\end{cases}
\end{equation}

and then we evolve the solution to the final $\U^{n+1}$ by solving the ``y - sweep'' over the full time interval $\Delta t$ :

\begin{equation}
	\begin{cases}
		\text{PDE: } & \deldt{\U} + \frac{\del}{\del y} \mathbf{G}(\U) = 0\\
		\text{IC: } & \U^{n+\half} \\
	\end{cases}
\end{equation}

using the 1D methods that are described.
This is called ``Strang splitting'' and gives us a first order accurate method.

For a second order accurate method, we need to do one more step in 2D:

\begin{enumerate}
	\item Get $\U^{n + 1/4}$ by doing an x - sweep over the time interval $\Delta t / 2$
	\item Get $\U^{n + 3/4}$ by doing an y - sweep over the time interval $\Delta t$ and using the intermediate solution $\U^{n + 1/4}$
	\item Get $\U^{n + 1}$ by doing an x - sweep over the time interval $\Delta t / 2$ and using the intermediate solution $\U^{n + 3/4}$
\end{enumerate}



However, if $\Delta t$ is kept constant over multiple time steps, which is often the case for linear advection, we can get the first order method described above to be second order accurate by switching the order of the sweeps every time step (see \cite{leveque_2002}), i.e.
if for one time step we did the x-sweep first, in the next one we do the y-sweep first, then in the next time step we do the x-sweep first again, and so on.
The order of the sweeps doesn't matter as long as all sweeps have been done in the end, and by alternating the order we recover the second order method formalism, but with a twice as big time step.
















