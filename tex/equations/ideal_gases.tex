\newpage
%=======================================
\section{Ideal Gases}
%=======================================





%=======================================
\subsection{Governing Equations}
%=======================================

We are mostly going to concern ourselves with ideal gases, which are described by the Euler equations:


\begin{align}
	\deldt
	\begin{pmatrix}
		\rho \\
		\rho \V \\
		E
	\end{pmatrix}
	+
	\nabla \cdot
	\begin{pmatrix}
		\rho \V\\
		\rho \V \otimes \V + p\\
		(E + p) \V
	\end{pmatrix}
	=
	\begin{pmatrix}
		0\\
		\rho \mathbf{a}\\
		\rho \mathbf{a} \V
	\end{pmatrix}
\end{align}



Where
\begin{itemize}
	\item $\rho$: fluid density
	\item $\V$: fluid (mean/bulk) velocity at a given point. I use the notation $\V = (u, v, w)$, or when indices are useful, $\V = (v_1, v_2, v_3)$
	\item $p$: pressure
	\item $E$: specific energy. $E = \frac{1}{2} \rho \V^2 + \rho \varepsilon$, with $\varepsilon$ = specific internal thermal energy
	\item $\mathbf{a}$: acceleration due to some external force.
\end{itemize}

The outer product $\cdot \otimes \cdot$ gives the following tensor:
\begin{align}
	(\V \otimes \V)_{ij} = \V_i \V_j
\end{align}



Furthermore, we have the following relations for ideal gasses:
\begin{align}
	p &= n k T \\
	p &= C \rho ^ \gamma && \text{entropy relation for smooth flow, i.e. no shocks} \\
	s &= c_V \ln \left( \frac{p}{\rho^\gamma} \right) + s_0 && \text{entropy}\\
	c &= \sqrt{\frac{\del p}{\del \rho} \bigg{|}_s } = \sqrt{\frac{\gamma p}{\rho}} && \text{sound speed}
\end{align}

with 
\begin{itemize}
	\item $n$ : number density
	\item $k$ : Boltzmann constant
	\item $T$ : temperature
	\item $s$ : entropy
	\item $\gamma$: adiabatic index
	\item $c_V$: specific heat 
\end{itemize}

and the Equation of State

\begin{align}
	\varepsilon &= \frac{1}{\gamma - 1}\frac{p}{\rho}
\end{align}




The Euler equations can be written as a conservation law as

\begin{align}
    \DELDT{ \U } + \nabla \cdot \F(\U) = 0
\end{align}

where we neglect any outer forces, i.e. $\mathbf{a} = 0$, and 


\begin{align}
	\U = 
		\begin{pmatrix}
			\rho \\ \rho \V \\ E
		\end{pmatrix}, &&
	\F(\U) = 
		\begin{pmatrix}
			\rho \V\\
			\rho \V \otimes \V + p\\
			(E + p) \V
		\end{pmatrix}
\end{align}









%===============================================
\subsubsection{Euler equations in 1D}
%===============================================

In 1D, we can write the Euler equations without source terms ($\mathbf{a} = 0$) as 

\begin{align}
    \DELDT{ \U } + \DELDX{\F(\U)} = 0
\end{align}

or explicitly (remember $\V = (u, v, w)$)


\begin{align}
	\deldt{ 
		\begin{pmatrix}
			\rho \\ \rho u \\ E
		\end{pmatrix}
		}
	+ \deldx {
		\begin{pmatrix}
			\rho u\\
			\rho u^2  + p\\
			(E + p) u
		\end{pmatrix}
	} = 0
\end{align}




%===============================================
\subsubsection{Euler equations in 2D}
%===============================================

In 2D, we have without source terms ($\mathbf{a} = 0$)



\begin{align}
    \DELDT{ \U } + \DELDX{\F(\U)} + \frac{\del \mathbf{G}(\U)}{\del y} = 0
\end{align}

or explicitly  (remember $\V = (u, v, w)$)

\begin{align}
	\deldt{ 
		\begin{pmatrix}
			\rho \\ \rho u \\ \rho v \\ E
		\end{pmatrix}
		}
	+ \deldx {
		\begin{pmatrix}
			\rho u\\
			\rho u^2  + p\\
			\rho uv\\
			(E + p) u
		\end{pmatrix}
	} 
	+ \frac{\del}{\del y}
		\begin{pmatrix}
			\rho v\\
			\rho uv\\
			\rho v^2  + p\\
			(E + p) v
		\end{pmatrix}
	= 0
\end{align}















%===============================================
\subsection{Conserved and primitive variables}
%===============================================


For now, we have described the Euler equation as a hyperbolic conservation law using the (conserved) state vector $\U$:

\begin{align}
	\U = 
		\begin{pmatrix}
			\rho \\ \rho \V \\ E
		\end{pmatrix}
\end{align}


for this reason, the variables $\rho$, $\rho \V$, and $E$ are referred to as ``conserved variables'', as they obey conservation laws.

However, this is not the only set of variables that allows us to describe the fluid dynamics.
In particular, the solution of the  Riemann problem (section \ref{chap:riemann}) will give us a set of with so called ``primitive variables'' (or ``physical variables'') with the ``primitive'' state vector $\mathbf{W}$

\begin{align}
	\mathbf{W} = 
		\begin{pmatrix}
			\rho \\ \V \\ p
		\end{pmatrix}
\end{align}



Using the ideal gas equations, these are the equations to translate between primitive and conservative variables:

\textbf{Primitive to conservative:}

\begin{align}
	(\rho) &= (\rho) \\
	(\rho \V) &= (\rho) \cdot (\V) \\
	(E) &= \frac{1}{2} (\rho) (\V)^2 + \frac{1}{\gamma - 1} \frac{(p)}{(\rho)}
\end{align}


\textbf{Conservative to primitive:}

\begin{align}
	(\rho) &= (\rho) \\
	(\V) &= \frac{(\rho \V )}{(\rho)}\\
	(p) &= \frac{\gamma - 1}{(\rho)}  \left((E) - \frac{1}{2} \frac{(\rho \V)^2}{(\rho)} \right) 
\end{align}













%=======================================
\subsection{Implementation Details}
%=======================================

All the functions for computing gas related quantities are written in \verb|gas.h| and \verb|gas.c|.
Every cell is represented by a struct \verb|struct cell| written in \verb|cell.h|.
It stores both the \underline{p}rimitive variables/states and the \underline{c}onservative states in the \texttt{\underline{p}state} and \texttt{\underline{c}state} structs, respectively.

The adiabatic index $\gamma$ is hardcoded as a macro in \texttt{defines.h}.
If you change it, all the derived quantities stored in macros (e.g. $\gamma - 1$, $\frac{1}{\gamma - 1}$) should be computed automatically.

